\documentclass[fontsize = 10pt,DIV = 13]{scrartcl}


\usepackage{amsmath}
\usepackage[noabbrev]{cleveref}
\usepackage{comment}
\usepackage{minted}
\usepackage{mdframed}  
\usepackage{xcolor}

\title{Direct solvers for dense matrices}


\newcommand{\pth}[1]{\left(#1\right)}
\surroundwithmdframed[linewidth=0pt, backgroundcolor=black!10]{minted} 

\begin{document}

\maketitle

\begin{abstract}
No abstract.
\end{abstract}
\newpage
\tableofcontents
\newpage

\section{Introduction}

Physical simulations and collision detection algorithms often need to solve linear equation systems of the form

\begin{align}
\label{eq:linearSystem}
\mathbf{A}\mathbf{x}=\mathbf{r}
\end{align}

$\mathbf{A}$ is known matrix of size $\mathrm{N}\times\mathrm{N}$ and $\mathbf{r}$ a known column vector of size $\mathrm{N}$.
The column vector $\mathbf{x}$ is also of size $\mathrm{N}$ but its values are unknown.

 


The subject of this document is how  unknown values of $\mathbf{x}$ can be determined. 
Different solvers and their efficient implementation in C++ using SSE and AVX instructions are discussed in \cref{sec:cramersRule,sec:gaussianElimination,sec:LU,sec:LDL,sec:choleskyDecomposition}.
The performance of each method depends on parameters like the system size $\mathrm{N}$, the symmetry of $\mathbf{A}$ and the number of solves with different $\mathbf{r}$ but identical $\mathbf{A}$.
Therefore, \cref{sec:solverSelection} discusses when which solver should be utilized to obtain maximal performance.

\begin{comment}


\subsection{Notation}

This document uses bold symbols for matrices and vectors.
Bold capital letters like $\mathbf{A}$ are used for matrices.
Lower case bold letters are used for vectors ($\mathbf{r}$, $\mathbf{x}$).
The individual elements of a vector are given by:

\begin{align*}
\mathbf{x}
=
\begin{bmatrix}
x_0\\
x_1\\
\vdots\\
x_{N-1}
\end{bmatrix}
\end{align*}

Zero based indexing is used since this makes it easier to translate the equations into C++ code.


The individual components of the matrix $\mathbf{A}$ can be distinguished by a single column specific character and a zero based row index.
The column specific characters start with the character $a$ for the first column and increase alphabetically.
For example, the components of $\mathbf{A}$ with size $3 \times 3$ are:

\begin{align*}
\mathbf{A}
=
\begin{bmatrix}
a_0&b_0&c_0\\
a_1&b_1&c_1\\
a_2&b_2&c_2
\end{bmatrix}
\end{align*}

This element naming is chosen because it can be easier translated into vectorized code than the common double index notation.
\end{comment}



\newpage
\section{Cramers Rule}
\label{sec:cramersRule}
\subsection{Theory}

Consider a linear system as given by \cref{eq:linearSystem} with a size of $\mathrm{N}=3$ and with


\begin{align*}
\mathbf{A}
=
\begin{bmatrix}
a_0&b_0&c_0\\
a_1&b_1&c_1\\
a_2&b_2&c_2
\end{bmatrix}
&&
\mathbf{x}
=
\begin{bmatrix}
x_0\\
x_1\\
x_2
\end{bmatrix}
&&
\mathbf{r}
=
\begin{bmatrix}
r_0\\
r_1\\
r_2
\end{bmatrix}
\end{align*}

The value $\mathrm{x}_i$ of the solution vector $\mathbf{x}$ can be determined by

\begin{align*}
\mathrm{x}_i = \frac{\mathrm{det}\pth{\mathbf{M}_i}}{\mathrm{det}\pth{\mathbf{A}}}
\end{align*}

The matrix $\mathbf{M}_i$ is obtained by replacing the i-th  column of $\mathbf{A}$ with the right-hand side vector $\mathbf{r}$.
For the given system of size  $\mathrm{N}=3$ the whole solution is computed as follows:

\begin{align*}
\mathbf{x}
=
\begin{bmatrix}
x_0\\
x_1\\
x_2
\end{bmatrix}
=
\begin{bmatrix}
\mathrm{det}\pth{\mathbf{M}_0} / \mathrm{det}\pth{\mathbf{A}}\\
\mathrm{det}\pth{\mathbf{M}_1} / \mathrm{det}\pth{\mathbf{A}}\\
\mathrm{det}\pth{\mathbf{M}_2} / \mathrm{det}\pth{\mathbf{A}}\\
\end{bmatrix}
\end{align*}

The matrices $\mathbf{M}_0$, $\mathbf{M}_1$ and $\mathbf{M}_2$ are given by:

\begin{align*}
\mathbf{M}_0
=
\begin{bmatrix}
r_0&b_0&c_0\\
r_1&b_1&c_1\\
r_2&b_2&c_2
\end{bmatrix}
&&
\mathbf{M}_1
=
\begin{bmatrix}
a_0&r_0&c_0\\
a_1&r_1&c_1\\
a_2&r_2&c_2
\end{bmatrix}
&&
\mathbf{M}_2
=
\begin{bmatrix}
a_0&b_0&r_0\\
a_1&b_1&r_1\\
a_2&b_2&r_2
\end{bmatrix}
\end{align*}

In the special case of $\mathrm{det}\pth{\mathbf{A}} = 0$, the matrix is singular and the system does not have a unique solution.

The equations to calculate the determinants of $2 \times 2$, $3 \times 3$ and $4 \times 4$ matrices are presented in \cref {sec:determinant2x2,sec:determinant3x3,sec:determinant4x4}.
Since the efficiency of Cramers rule drops significantly for larger systems, other determinants are usually not needed. 




\subsubsection{2x2 determinant}
\label{sec:determinant2x2}

\begin{align*}
\mathbf{A}
=
\begin{bmatrix}
a_0&b_0\\
a_1&b_1\\
\end{bmatrix}
\end{align*}

\begin{align*}
\mathrm{det}\pth{\mathbf{A}}
&
= a_0b_1 - a_1b_0
\end{align*}

\subsubsection{3x3 determinant}
\label{sec:determinant3x3}
\begin{align*}
\mathbf{A}
=
\begin{bmatrix}
a_0&b_0&c_0\\
a_1&b_1&c_1\\
a_2&b_2&c_2
\end{bmatrix}
\end{align*}

\begin{align*}
\mathrm{det}\pth{\mathbf{A}}
&
= a_0b_1c_2 - a_0b_2c_1 
\\&
+ a_1b_2c_0 - a_1b_0c_2 
\\&
+ a_2b_0c_1 - a_2b_1c_0
\end{align*}

\begin{align*}
\mathrm{det}\pth{\mathbf{A}}
&
= a_0\pth{b_1c_2 - b_2c_1}
\\&
+ a_1\pth{b_2c_0 - b_0c_2}
\\&
+ a_2\pth{b_0c_1 - b_1c_0}
\end{align*}

If all three columns of $\mathbf{A}$ are treated as individual vectors $\mathbf{a}$, $\mathbf{b}$, $\mathbf{c}$ the determinant of $\mathbf{A}$ can be written as:

\begin{align}
\label{eq:determinant3x3}
\mathrm{det}\pth{\mathbf{A}}
&
= \mathbf{a} \cdot \pth{\mathbf{b} \times \mathbf{c}}
\end{align}

The determinant of  $\mathbf{A}$ is the dot product of the first column and the cross product of second and third column.

\subsubsection{4x4 determinant}
\label{sec:determinant4x4}
\begin{align*}
\mathbf{A}
=
\begin{bmatrix}
a_0&b_0&c_0&d_0\\
a_1&b_1&c_1&d_1\\
a_2&b_2&c_2&d_2\\
a_3&b_3&c_3&d_3
\end{bmatrix}
\end{align*}


\begin{comment}

\begin{align*}
det\pth{\mathbf{A}}
&=
a_1b_2c_3d_4 
+ a_1b_3c_4d_2 
+ a_1b_4c_2d_3 
- a_1b_4c_3d_2
- a_1b_2c_4d_3\\
&- a_1b_3c_2d_4
- a_2b_1c_3d_4
- a_4b_1c_2d_3
- a_3b_1c_4d_2
+ a_4b_1c_3d_2\\
&+ a_2b_1c_4d_3
+ a_3b_1c_2d_4
+ a_2b_3c_1d_4
+ a_4b_2c_1d_3
+ a_3b_4c_1d_2\\
&- a_4b_3c_1d_2
- a_2b_4c_1d_3
- a_3b_2c_1d_4
- a_2b_3c_4d_1
- a_4b_2c_3d_1\\
&- a_3b_4c_2d_1
+ a_4b_3c_2d_1
+ a_2b_4c_3d_1
+ a_3b_2c_4d_1           
\end{align*}

\end{comment}

The determinant of $\mathbf{A}$ can be calculated with the equation:

\begin{align}
\mathrm{det}\pth{\mathbf{A}}
&
= a_0b_1c_2d_3 
+ a_0b_2c_3d_1 
+ a_0b_3c_1d_2
- a_0b_1c_3d_2
- a_0b_2c_1d_3 
- a_0b_3c_2d_1\nonumber\\
&
+ a_1b_0c_3d_2
+ a_1b_2c_0d_3
+ a_1b_3c_2d_0
- a_1b_0c_2d_3
- a_1b_2c_3d_0
- a_1b_3c_0d_2\nonumber\\
&
+ a_2b_0c_1d_3
+ a_2b_1c_3d_0
+ a_2b_3c_0d_1
- a_2b_0c_3d_1
- a_2b_1c_0d_3
- a_2b_3c_1d_0\nonumber\\
&
+ a_3b_0c_2d_1
+ a_3b_1c_0d_2
+ a_3b_2c_1d_0
- a_3b_0c_1d_2
- a_3b_1c_2d_0
- a_3b_2c_0d_1\label{eq:determinant4x4}
\end{align}

\Cref{eq:determinant4x4} can be simplified as follows:

\begin{align*}
\mathrm{det}\pth{\mathbf{A}}
= a_0b_1 \pth{c_2d_3 - c_3d_2} 
+ a_0b_2 \pth{c_3d_1 - c_1d_3}
+ a_0b_3 \pth{c_1d_2 - c_2d_1}\\
+ a_1b_0 \pth{c_3d_2 - c_2d_3}
+ a_1b_2 \pth{c_0d_3 - c_3d_0}
+ a_1b_3 \pth{c_2d_0 - c_0d_2}\\
+ a_2b_0 \pth{c_1d_3 - c_3d_1}
+ a_2b_1 \pth{c_3d_0 - c_0d_3}
+ a_2b_3 \pth{c_0d_1 - c_1d_0}\\
+ a_3b_0 \pth{c_2d_1 - c_1d_2}
+ a_3b_1 \pth{c_0d_2 - c_2d_0}
+ a_3b_2 \pth{c_1d_0 - c_0d_1}
\end{align*}




\begin{align*}
\mathrm{det}\pth{\mathbf{A}}
= \pth{a_0b_1 - a_1b_0} \pth{c_2d_3 - c_3d_2}\\
+ \pth{a_0b_2 - a_2b_0} \pth{c_3d_1 - c_1d_3}\\
+ \pth{a_0b_3 - a_3b_0} \pth{c_1d_2 - c_2d_1}\\
+ \pth{a_1b_2 - a_2b_1} \pth{c_0d_3 - c_3d_0}\\
+ \pth{a_1b_3 - a_3b_1} \pth{c_2d_0 - c_0d_2}\\
+ \pth{a_2b_3 - a_3b_2} \pth{c_0d_1 - c_1d_0}\\
\end{align*}

The order of the terms between the parenthesis can be changed as follows:
\begin{align*}
\pth{u-v}\pth{x-y} 
&= \pth{-v+u}\pth{-y+x}\\
&= -1\cdot\pth{v-u}\cdot -1 \cdot \pth{y-x} \\
&= \pth{v-u}\pth{y-x} 
\end{align*} 

Using the previously shown relation and reordering the terms yields:

\begin{align}
\mathrm{det}\pth{\mathbf{A}}
= \pth{a_0b_1 - a_1b_0} \pth{c_2d_3 - c_3d_2}\nonumber\\
+ \pth{a_1b_2 - a_2b_1} \pth{c_0d_3 - c_3d_0}\nonumber\\
+ \pth{a_2b_3 - a_3b_2} \pth{c_0d_1 - c_1d_0}\nonumber\\
+ \pth{a_3b_0 - a_0b_3} \pth{c_2d_1 - c_1d_2}\nonumber\\
+ \pth{a_0b_2 - a_2b_0} \pth{c_3d_1 - c_1d_3}\nonumber\\
+ \pth{a_1b_3 - a_3b_1} \pth{c_2d_0 - c_0d_2}\label{eq:determinant4x4SSE}
\end{align}

This ordering is useful for the SSE implementation described in \cref{sec:determinant4x4SSE}. 
An alternative formulation, that can be derived the same way as \cref{eq:determinant4x4SSE}, but that is more appropriate for AVX registers is:   

\begin{comment}

\begin{align*}
a_0c_1 \pth{b_3d_2 - b_2d_3}
+ a_0c_2 \pth{b_1d_3 - b_3d_1}            
+ a_0c_3 \pth{b_2d_1 - b_1d_2}\\
+ a_1c_0 \pth{b_2d_3 - b_3d_2}
+ a_1c_2 \pth{b_3d_0 - b_0d_3}
+ a_1c_3 \pth{b_0d_2 - b_2d_0}\\
+ a_2c_0 \pth{b_3d_1 - b_1d_3}
+ a_2c_1 \pth{b_0d_3 - b_3d_0}
+ a_2c_3 \pth{b_1d_0 - b_0d_1}\\
+ a_3c_0 \pth{b_1d_2 - b_2d_1}
+ a_3c_1 \pth{b_2d_0 - b_0d_2}
+ a_3c_2 \pth{b_0d_1 - b_1d_0}
\end{align*}

\begin{align*}
\pth{a_0c_1 - a_1c_0} \pth{b_3d_2 - b_2d_3} \\
+ \pth{a_0c_2 - a_2c_0} \pth{b_1d_3 - b_3d_1} \\
+ \pth{a_0c_3 - a_3c_0} \pth{b_2d_1 - b_1d_2} \\
+ \pth{a_1c_2 - a_2c_1} \pth{b_3d_0 - b_0d_3} \\
+ \pth{a_1c_3 - a_3c_1} \pth{b_0d_2 - b_2d_0} \\ 
+ \pth{a_2c_3 - a_3c_2} \pth{b_1d_0 - b_0d_1} \\ 
\end{align*}
\end{comment}

\begin{align}
\mathrm{det}\pth{\mathbf{A}}
= \pth{a_0c_1 - a_1c_0} \pth{b_3d_2 - b_2d_3} \nonumber\\
+ \pth{a_1c_2 - a_2c_1} \pth{b_3d_0 - b_0d_3} \nonumber\\ 
+ \pth{a_2c_3 - a_3c_2} \pth{b_1d_0 - b_0d_1} \nonumber\\
+ \pth{a_3c_0 - a_0c_3} \pth{b_1d_2 - b_2d_1} \nonumber\\
+ \pth{a_0c_2 - a_2c_0} \pth{b_1d_3 - b_3d_1} \nonumber\\
+ \pth{a_1c_3 - a_3c_1} \pth{b_0d_2 - b_2d_0} \label{eq:determinant4x4AVX}
\end{align}




% ------------------------------------------------------------------------------
\newpage
\subsection{Implementation}
\subsubsection{3x3 determinant - serial}
\label{sec:determinant3x3Serial}

Since the serial calculation of a 3x3 determinant is just turning \cref{eq:determinant3x3} into C++ code, the function is not further discussed.
The code for a matrix with column major ordering is as follows:

\begin{minted}{cpp}
template <typename _type>
F32 Mat3Serial<_type>::Det() const
{
    return mD[0] * (mD[4] * mD[8] - mD[5] * mD[7]) + 
           mD[3] * (mD[7] * mD[2] - mD[8] * mD[1]) +
           mD[6] * (mD[1] * mD[5] - mD[2] * mD[4]);
}
\end{minted}




% ------------------------------------------------------------------------------
\newpage
\subsubsection{3x3 solver - serial}
\label{sec:cramerSolver3x3Serial}

The serial implementation of Cramers rule just calculates four determinants as presented in \cref{sec:determinant3x3Serial} with one cross product being reused.
Because of its simplicity, the code is presented without any further explanation.


\begin{minted}{cpp}
template <typename _type>
inline Vec3Serial<_type, true> Cramer(const Mat3Serial<_type>& matA, 
                                      const Vec3Serial<_type, true>& vecRhs)
{
    std::array<_type, 9> a = matA.Data();
    std::array<_type, 3> r = vecRhs.Data();

    std::array<_type, 3> crossbc = {{a[4] * a[8] - a[5] * a[7], 
                                     a[5] * a[6] - a[3] * a[8], 
                                     a[3] * a[7] - a[4] * a[6]}};

    std::array<_type, 3> crossrc = {{r[1] * a[8] - r[2] * a[7], 
                                     r[2] * a[6] - r[0] * a[8], 
                                     r[0] * a[7] - r[1] * a[6]}};

    std::array<_type, 3> crossbr = {{a[4] * r[2] - a[5] * r[1], 
                                     a[5] * r[0] - a[3] * r[2], 
                                     a[3] * r[1] - a[4] * r[0]}};

    _type detA = a[0] * crossbc[0] + a[1] * crossbc[1] + a[2] * crossbc[2];

    DEV_EXCEPTION(detA == ApproxZero<F32>(10), 
                          "Singular matrix - system not solveable");

    std::array<_type, 3> solution = 
        {{(r[0] * crossbc[0] + r[1] * crossbc[1] + r[2] * crossbc[2]) / detA,
          (a[0] * crossrc[0] + a[1] * crossrc[1] + a[2] * crossrc[2]) / detA,
          (a[0] * crossbr[0] + a[1] * crossbr[1] + a[2] * crossbr[2]) / detA}};

    return Vec3Serial<_type, true>(solution);
}
\end{minted}


% ------------------------------------------------------------------------------
\newpage
\subsubsection{3x3 determinant - SSE}
\label{sec:determinant3x3SSE}

The matrix

\begin{align*}
\mathbf{A}
=
\begin{bmatrix}
a_0&b_0&c_0\\
a_1&b_1&c_1\\
a_2&b_2&c_2
\end{bmatrix}
\end{align*}

is represented by three \mintinline{cpp}{__m128} registers:

\begin{align*}
\mathrm{a} 
=
\begin{bmatrix}
a_0\\
a_1\\
a_2\\
0\\
\end{bmatrix}
&&
\mathrm{b} 
&=
\begin{bmatrix}
b_0\\
b_1\\
b_2\\
0\\
\end{bmatrix}
&
\mathrm{c} 
=
\begin{bmatrix}
c_0\\
c_1\\
c_2\\
0\\
\end{bmatrix}
\end{align*}

In accordance with \cref{eq:determinant3x3}, the implementation of the 3x3 determinant can be written in a single row:

\begin{minted}{cpp}
inline F32 Determinant3x3(__m128 a, __m128 b, __m128 c)
{
    return DotProductF32<1, 1, 1, 0>(a, CrossProduct(b, c));
}
\end{minted}

\mintinline{cpp}{DotProductF32} uses the intrinsic function \mintinline{cpp}{ _mm_dp_ps} with a fitting mask so that only the first three values of each register are used during the calculation of the dot product. 
The cross product is implemented as follows:

\begin{minted}{cpp}
[[nodiscard]] inline __m128 CrossProduct(__m128 lhs, __m128 rhs)
{
    __m128 lhs_yzx = Permute<1, 2, 0, 3>(lhs);
    __m128 rhs_yzx = Permute<1, 2, 0, 3>(rhs);

    __m128 tmp = _mm_fmsub(lhs, rhs_yzx, _mm_mul(lhs_yzx, rhs));

    return Permute<1, 2, 0, 3>(tmp);
}
\end{minted}

Because this is a rather simple function, it won't be explained any further.



% ------------------------------------------------------------------------------
\newpage
\subsubsection{3x3 solver - SSE}
\label{sec:cramerSolver3x3SSE}

First, the determinant of the unmodified matrix is calculated. Since the cross product of the second and third column can be reused later, it is stored in an extra variable.

\begin{minted}{cpp}
__m128 crossbc = CrossProduct(b, c);
__m128 detA = DotProduct<1, 1, 1, 0>(a, crossbc);
\end{minted}

Here \mintinline{cpp}{DotProduct} is used instead of \mintinline{cpp}{DotProductF32}, which was used in the determinant calculation of \cref{sec:determinant3x3SSE}. 
The difference between both functions is that \mintinline{cpp}{DotProduct} returns a \mintinline{cpp}{__m128} register with all values being equal to the dot product instead of a \mintinline{cpp}{F32}.

Before the modified determinants are calculated, a check is performed, if the system is solvable.

\begin{minted}{cpp}
DEV_EXCEPTION(_mm_cvtsF(detA) == ApproxZero<F32>(10), 
              "Singular matrix - system not solveable");
\end{minted}

Now the determinants of all modified matrices are calculated and blended into their corresponding position:

\begin{minted}{cpp}
__m128 crossrc = CrossProduct(r, c);
__m128 crossbr = CrossProduct(b, r);

__m128 determinants = DotProduct<1, 1, 1, 0>(r, crossbc);
determinants = Blend<0, 1, 0, 0>(determinants, DotProduct<1, 1, 1, 0>(a, crossrc));
determinants = Blend<0, 0, 1, 0>(determinants, DotProduct<1, 1, 1, 0>(a, crossbr));
\end{minted}

Finally, the modified determinants are divided by the determinant of the unmodified matrix to get the system's solution:

\begin{minted}{cpp}
__m128 solution = _mmx_div_p(determinants, detA);

return Vec3fSSE<true>(solution);
\end{minted}


\vspace{1cm}
\textbf{Complete Function:}

\begin{minted}{cpp}
inline Vec3fSSE<true> Cramer(const Mat3fSSE& matA, const Vec3fSSE<true>& vecRhs)
{
    using namespace GDL::simd;

    const std::array<__m128, 3>& dataA = matA.DataSSE();
    const __m128& a = dataA[0];
    const __m128& b = dataA[1];
    const __m128& c = dataA[2];

    const __m128& r = vecRhs.DataSSE();

    __m128 crossbc = CrossProduct(b, c);
    __m128 detA = DotProduct<1, 1, 1, 0>(a, crossbc);

    DEV_EXCEPTION(_mm_cvtsF(detA) == ApproxZero<F32>(10), 
                  "Singular matrix - system not solveable");

    __m128 crossrc = CrossProduct(r, c);
    __m128 crossbr = CrossProduct(b, r);

    __m128 determinants = DotProduct<1, 1, 1, 0>(r, crossbc);
    determinants = 
        Blend<0, 1, 0, 0>(determinants, DotProduct<1, 1, 1, 0>(a, crossrc));
    determinants = 
        Blend<0, 0, 1, 0>(determinants, DotProduct<1, 1, 1, 0>(a, crossbr));

    __m128 solution = _mm_div(determinants, detA);

    return Vec3fSSE<true>(solution);
}
\end{minted}




% ------------------------------------------------------------------------------
\newpage
\subsubsection{4x4 determinant - serial}
\label{sec:determinant4x4Serial}

Since the serial calculation of a 4x4 determinant is just turning \cref{eq:determinant4x4SSE} into C++ code, the function is not further discussed.
The code for a matrix with column major ordering is as follows:

\begin{minted}{cpp}
F32 Mat4Serial<_type>::Det() const
{
    F32 ab01 = mD[0]  * mD[5]  - mD[1]  * mD[4];
    F32 cd23 = mD[10] * mD[15] - mD[11] * mD[14];

    F32 ab12 = mD[1]  * mD[6]  - mD[2]  * mD[5];
    F32 cd03 = mD[8]  * mD[15] - mD[11] * mD[12];

    F32 ab23 = mD[2]  * mD[7]  - mD[3]  * mD[6];
    F32 cd01 = mD[8]  * mD[13] - mD[9]  * mD[12];

    F32 ab30 = mD[3]  * mD[4]  - mD[0]  * mD[7];
    F32 cd21 = mD[10] * mD[13] - mD[9]  * mD[14];

    F32 ab02 = mD[0]  * mD[6]  - mD[2]  * mD[4];
    F32 cd31 = mD[11] * mD[13] - mD[9]  * mD[15];

    F32 ab13 = mD[1]  * mD[7]  - mD[3]  * mD[5];
    F32 cd20 = mD[10] * mD[12] - mD[8]  * mD[14];

    return ab01 * cd23 + ab12 * cd03 + ab23 * cd01 + 
           ab30 * cd21 + ab02 * cd31 + ab13 * cd20;
}
\end{minted}





% ------------------------------------------------------------------------------
\newpage
\subsubsection{4x4 solver - serial}
\label{sec:cramerSolver4x4Serial}

The serial implementation of Cramers rule just calculates five determinants as presented in \cref{sec:determinant4x4Serial} with some intermediate results being reused.
Because of its simplicity, the code is presented without any further explanation.

\begin{minted}{cpp}
template <typename _type>
inline Vec4Serial<_type, true> Cramer(const Mat4Serial<_type>& matA, 
                                      const Vec4Serial<_type, true>& vecRhs)
{
    const std::array<_type, 16>& A = matA.Data();


    // Calculate detrminant of A
    F32 ab01 = A[0]  * A[5]  - A[1]  * A[4];
    F32 cd23 = A[10] * A[15] - A[11] * A[14];

    F32 ab12 = A[1]  * A[6]  - A[2]  * A[5];
    F32 cd03 = A[8]  * A[15] - A[11] * A[12];

    F32 ab23 = A[2]  * A[7]  - A[3]  * A[6];
    F32 cd01 = A[8]  * A[13] - A[9]  * A[12];

    F32 ab30 = A[3]  * A[4]  - A[0]  * A[7];
    F32 cd21 = A[10] * A[13] - A[9]  * A[14];

    F32 ab02 = A[0]  * A[6]  - A[2]  * A[4];
    F32 cd31 = A[11] * A[13] - A[9]  * A[15];

    F32 ab13 = A[1]  * A[7]  - A[3]  * A[5];
    F32 cd20 = A[10] * A[12] - A[8]  * A[14];

    _type detA = ab01 * cd23 + ab12 * cd03 + ab23 * cd01 + 
                 ab30 * cd21 + ab02 * cd31 + ab13 * cd20;


    DEV_EXCEPTION(detA == ApproxZero<F32>(10), 
                  "Singular matrix - system not solveable");


    const std::array<_type, 4>& r = vecRhs.Data();


    // Calculate solution with modified determinants
    F32 rb01 = r[0] * A[5] - r[1] * A[4];
    F32 rb12 = r[1] * A[6] - r[2] * A[5];
    F32 rb23 = r[2] * A[7] - r[3] * A[6];
    F32 rb30 = r[3] * A[4] - r[0] * A[7];
    F32 rb02 = r[0] * A[6] - r[2] * A[4];
    F32 rb13 = r[1] * A[7] - r[3] * A[5];

    std::array<_type, 4> solution;
    solution[0] = (rb01 * cd23 + rb12 * cd03 + rb23 * cd01 + 
                   rb30 * cd21 + rb02 * cd31 + rb13 * cd20) / detA;


    F32 ar01 = A[0] * r[1] - A[1] * r[0];
    F32 ar12 = A[1] * r[2] - A[2] * r[1];
    F32 ar23 = A[2] * r[3] - A[3] * r[2];
    F32 ar30 = A[3] * r[0] - A[0] * r[3];
    F32 ar02 = A[0] * r[2] - A[2] * r[0];
    F32 ar13 = A[1] * r[3] - A[3] * r[1];

    solution[1] = (ar01 * cd23 + ar12 * cd03 + ar23 * cd01 + 
                   ar30 * cd21 + ar02 * cd31 + ar13 * cd20) / detA;


    F32 rd23 = r[2] * A[15] - r[3] * A[14];
    F32 rd03 = r[0] * A[15] - r[3] * A[12];
    F32 rd01 = r[0] * A[13] - r[1] * A[12];
    F32 rd21 = r[2] * A[13] - r[1] * A[14];
    F32 rd31 = r[3] * A[13] - r[1] * A[15];
    F32 rd20 = r[2] * A[12] - r[0] * A[14];

    solution[2] = (ab01 * rd23 + ab12 * rd03 + ab23 * rd01 + 
                   ab30 * rd21 + ab02 * rd31 + ab13 * rd20) / detA;


    F32 cr23 = A[10] * r[3] - A[11] * r[2];
    F32 cr03 = A[8]  * r[3] - A[11] * r[0];
    F32 cr01 = A[8]  * r[1] - A[9]  * r[0];
    F32 cr21 = A[10] * r[1] - A[9]  * r[2];
    F32 cr31 = A[11] * r[1] - A[9]  * r[3];
    F32 cr20 = A[10] * r[0] - A[8]  * r[2];

    solution[3] = (ab01 * cr23 + ab12 * cr03 + ab23 * cr01 + 
                   ab30 * cr21 + ab02 * cr31 + ab13 * cr20) / detA;


    return Vec4Serial<_type, true>(solution);
}
\end{minted}




% ------------------------------------------------------------------------------
\newpage
\subsubsection{4x4 determinant - SSE}
\label{sec:determinant4x4SSE}

Each column of the matrix

\begin{align*}
\mathbf{A}
=
\begin{bmatrix}
a_0&b_0&c_0&d_0\\
a_1&b_1&c_1&d_1\\
a_2&b_2&c_2&d_2\\
a_3&b_3&c_3&d_3
\end{bmatrix}
\end{align*}

is stored in a \mintinline{cpp}{__m128} register:

\begin{align*}
\mathrm{a} 
=
\begin{bmatrix}
a_0\\
a_1\\
a_2\\
a_3\\
\end{bmatrix}
&&
\mathrm{b} 
&=
\begin{bmatrix}
b_0\\
b_1\\
b_2\\
b_3\\
\end{bmatrix}
&
\mathrm{c} 
=
\begin{bmatrix}
c_0\\
c_1\\
c_2\\
c_3\\
\end{bmatrix}
&&
\mathrm{d} 
&=
\begin{bmatrix}
d_0\\
d_1\\
d_2\\
d_3\\
\end{bmatrix}
\end{align*}



The SSE algorithm to calculate the determinant is based on \cref{eq:determinant4x4SSE}.
Since it only needs to be translated into a minimal set of SSE instructions and no complicated operations are involved, there won't be a detailed explanation of each line of code.
Instead, the content of the affected variables is shown after each code segment.

The algorithm starts by calculating the first 4 products of \cref{eq:determinant4x4SSE}.


\begin{minted}{cpp}
__m128 aP1230 = Permute<1, 2, 3, 0>(a);
__m128 bP1230 = Permute<1, 2, 3, 0>(b);
__m128 cP1230 = Permute<1, 2, 3, 0>(c);
__m128 dP1230 = Permute<1, 2, 3, 0>(d);
\end{minted}

\begin{align*}
\mathrm{aP1230} 
=
\begin{bmatrix}
a_1\\
a_2\\
a_3\\
a_0\\
\end{bmatrix}
&&
\mathrm{bP1230} 
&=
\begin{bmatrix}
b_1\\
b_2\\
b_3\\
b_0\\
\end{bmatrix}
&
\mathrm{cP1230} 
=
\begin{bmatrix}
c_1\\
c_2\\
c_3\\
c_0\\
\end{bmatrix}
&&
\mathrm{dP1230} 
&=
\begin{bmatrix}
d_1\\
d_2\\
d_3\\
d_0\\
\end{bmatrix}
\end{align*}

\begin{minted}{cpp}
__m128 ab03 = _mm_fmsub(a, bP1230, _mm_mul(aP1230, b));
__m128 cd03 = _mm_fmsub(c, dP1230, _mm_mul(cP1230, d));
\end{minted}


\begin{align*}
\mathrm{ab03} 
=
\begin{bmatrix}
a_0b_1 - a_1b_0\\
a_1b_2 - a_2b_1\\
a_2b_3 - a_3b_2\\
a_3b_0 - a_0b_3\\
\end{bmatrix}
&&
\mathrm{cd03} 
&=
\begin{bmatrix}
c_0d_1 - c_1d_0\\
c_1d_2 - c_2d_1\\
c_2d_3 - c_3d_2\\
c_3d_0 - c_0d_3\\
\end{bmatrix}
\end{align*}

\begin{minted}{cpp}
__m128 cd03P2301 = Permute<2, 3, 0, 1>(cd03);
\end{minted}

\begin{align*}
\mathrm{cd03P2301} 
=
\begin{bmatrix}
c_2d_3 - c_3d_2\\
c_3d_0 - c_0d_3\\
c_0d_1 - c_1d_0\\
c_1d_2 - c_2d_1\\
\end{bmatrix}
\end{align*}

\begin{minted}{cpp}
__m128 cd03P2301N = Negate<0, 1, 0, 1>(cd03P2301);
\end{minted}

\begin{align*}
\mathrm{cd03P2301N} 
=
\begin{bmatrix}
c_2d_3 - c_3d_2\\
c_0d_3 - c_3d_0\\
c_0d_1 - c_1d_0\\
c_2d_1 - c_1d_2\\
\end{bmatrix}
\end{align*}

\begin{minted}{cpp}
__m128 sum03 = DotProduct(ab03, cd03P2301N);
\end{minted}

\begin{align*}
\mathrm{sum03} 
&=
\begin{bmatrix}
\mathrm{s03}\\
\mathrm{s03}\\
\mathrm{s03}\\
\mathrm{s03}\\
\end{bmatrix}\\
\\
\mathrm{s03}
&= \pth{a_0b_1 - a_1b_0} \pth{c_2d_3 - c_3d_2}\\
&+ \pth{a_1b_2 - a_2b_1} \pth{c_0d_3 - c_3d_0}\\
&+ \pth{a_2b_3 - a_3b_2} \pth{c_0d_1 - c_1d_0}\\
&+ \pth{a_3b_0 - a_0b_3} \pth{c_2d_1 - c_1d_2}\\
\end{align*}


Now the last 2 products of \cref{eq:determinant4x4SSE} are calculated.

\begin{minted}{cpp}
__m128 acB0011 = Blend<0, 0, 1, 1>(a, c);
__m128 acB1100 = Blend<1, 1, 0, 0>(a, c);
__m128 bdB0011 = Blend<0, 0, 1, 1>(b, d);
__m128 bdB1100 = Blend<1, 1, 0, 0>(b, d);
\end{minted}

\begin{align*}
\mathrm{acB0011} 
=
\begin{bmatrix}
a_0\\
a_1\\
c_2\\
c_3\\
\end{bmatrix}
&&
\mathrm{acB1100} 
&=
\begin{bmatrix}
c_0\\
c_1\\
a_2\\
a_3\\
\end{bmatrix}
&
\mathrm{bdB0011} 
=
\begin{bmatrix}
b_0\\
b_1\\
d_2\\
d_3\\
\end{bmatrix}
&&
\mathrm{bdB1100} 
&=
\begin{bmatrix}
d_0\\
d_1\\
b_2\\
b_3\\
\end{bmatrix}
\end{align*}

\begin{minted}{cpp}
__m128 acB1100P2301 = Permute<2, 3, 0, 1>(acB1100);
__m128 bdB1100P2301 = Permute<2, 3, 0, 1>(bdB1100);
\end{minted}

\begin{align*}
\mathrm{acB1100P2301} 
=
\begin{bmatrix}
a_2\\
a_3\\
c_0\\
c_1\\
\end{bmatrix}
&&
\mathrm{bdB1100P2301} 
&=
\begin{bmatrix}
b_2\\
b_3\\
d_0\\
d_1\\
\end{bmatrix}
\end{align*}


\begin{minted}{cpp}
__m128 abcd45 = _mm_fmsub(acB0011, bdB1100P2301, _mm_mul(acB1100P2301, bdB0011));
\end{minted}

\begin{align*}
\mathrm{abcd45} 
=
\begin{bmatrix}
a_0b_2 - a_2b_0\\
a_1b_3 - a_3b_1\\
c_2d_0 - c_0d_2\\
c_3d_1 - c_1d_3\\
\end{bmatrix}
\end{align*}

\begin{minted}{cpp}
__m128 abcd45P3210 = Permute<3, 2, 1, 0>(abcd45);
\end{minted}

\begin{align*}
\mathrm{abcd45P3210} 
=
\begin{bmatrix}
c_3d_1 - c_1d_3\\
c_2d_0 - c_0d_2\\
a_1b_3 - a_3b_1\\
a_0b_2 - a_2b_0\\
\end{bmatrix}
\end{align*}

\begin{minted}{cpp}
__m128 products45 = _mm_mul(abcd45, abcd45P3210);
\end{minted}

\begin{align*}
\mathrm{products45} 
=
\begin{bmatrix}
\pth{a_0b_2 - a_2b_0}\pth{c_3d_1 - c_1d_3}\\
\pth{a_1b_3 - a_3b_1}\pth{c_2d_0 - c_0d_2}\\
\pth{a_1b_3 - a_3b_1}\pth{c_2d_0 - c_0d_2}\\
\pth{a_0b_2 - a_2b_0}\pth{c_3d_1 - c_1d_3}\\
\end{bmatrix}
\end{align*}


\begin{minted}{cpp}
__m128 sum45 = _mm_add(products45, Permute<1, 0, 3, 2>(products45));
\end{minted}



\begin{align*}
\mathrm{sum45} 
=
\begin{bmatrix}
  \pth{a_0b_2 - a_2b_0}\pth{c_3d_1 - c_1d_3}
+ \pth{a_1b_3 - a_3b_1}\pth{c_2d_0 - c_0d_2}\\
  \pth{a_0b_2 - a_2b_0}\pth{c_3d_1 - c_1d_3}
+ \pth{a_1b_3 - a_3b_1}\pth{c_2d_0 - c_0d_2}\\
  \pth{a_0b_2 - a_2b_0}\pth{c_3d_1 - c_1d_3}
+ \pth{a_1b_3 - a_3b_1}\pth{c_2d_0 - c_0d_2}\\
  \pth{a_0b_2 - a_2b_0}\pth{c_3d_1 - c_1d_3}
+ \pth{a_1b_3 - a_3b_1}\pth{c_2d_0 - c_0d_2}\\
\end{bmatrix}
\end{align*}

Finally, the determinant can be returned:

\begin{minted}{cpp}
return _mm_cvtsF(sum03) + _mm_cvtsF(sum45);
\end{minted}




\vspace{1cm}
\textbf{Complete Function:}

\begin{minted}{cpp}
inline F32 Determinant4x4(__m128 a, __m128 b, __m128 c, __m128 d)
{
    // Calculate sum of first 4 terms
    __m128 aP1230 = Permute<1, 2, 3, 0>(a);
    __m128 bP1230 = Permute<1, 2, 3, 0>(b);
    __m128 cP1230 = Permute<1, 2, 3, 0>(c);
    __m128 dP1230 = Permute<1, 2, 3, 0>(d);

    __m128 ab03 = _mm_fmsub(a, bP1230, _mm_mul(aP1230, b));
    __m128 cd03 = _mm_fmsub(c, dP1230, _mm_mul(cP1230, d));

    __m128 cd03P2301 = Permute<2, 3, 0, 1>(cd03);
    __m128 cd03P2301N = Negate<0, 1, 0, 1>(cd03P2301);

    __m128 sum03 = DotProduct(ab03, cd03P2301N);


    // Calculate sum of last 2 terms
    __m128 acB0011 = Blend<0, 0, 1, 1>(a, c);
    __m128 acB1100 = Blend<1, 1, 0, 0>(a, c);
    __m128 bdB0011 = Blend<0, 0, 1, 1>(b, d);
    __m128 bdB1100 = Blend<1, 1, 0, 0>(b, d);

    __m128 acB1100P2301 = Permute<2, 3, 0, 1>(acB1100);
    __m128 bdB1100P2301 = Permute<2, 3, 0, 1>(bdB1100);

    __m128 abcd45 = _mm_fmsub(acB0011, bdB1100P2301, _mm_mul(acB1100P2301, bdB0011));
    __m128 abcd45P3210 = Permute<3, 2, 1, 0>(abcd45);

    __m128 products45 = _mm_mul(abcd45, abcd45P3210);
    __m128 sum45 = _mm_add(products45, Permute<1, 0, 3, 2>(products45));


    // Calculate and return determinant
    return _mm_cvtsF(sum03) + _mm_cvtsF(sum45);
}
\end{minted}




% ------------------------------------------------------------------------------
\newpage
\subsubsection{4x4 solver - SSE}

First the determinant of $\mathbf{A}$ is calculated.
Because some intermediate results can be reused, the previously introduced determinant function of \cref{sec:determinant4x4SSE} is not employed here. 
Instead, its code is copied. 
The resulting code duplication can be justified by the huge performance gain.

The return statement from the original code is removed and a register with all values set to the determinant of $\mathbf{A}$ is created:

\begin{minted}{cpp}
__m128 detA = _mm_add(sum03, sum45);
\end{minted}

Afterwards, the following line ensures that the system is solvable:

\begin{minted}{cpp}
DEV_EXCEPTION(_mm_cvtsF(detA) == ApproxZero<F32>(10), 
              "Singular matrix - system not solveable");
\end{minted} 

The right-hand side vector $\mathbf{r}$ is stored in a single \mintinline{cpp}{__m128} register:


\begin{align*}
\mathrm{r} 
=
\begin{bmatrix}
r_0\\
r_1\\
r_2\\
r_3\\
\end{bmatrix}
\end{align*}


Then the missing values to calculate the first 4 terms of \cref{eq:determinant4x4SSE} for the modified determinants are obtained by the following instructions:

\begin{minted}{cpp}
__m128 rP1230 = Permute<1, 2, 3, 0>(r);
\end{minted}


\begin{align*}
\mathrm{rP1230} 
=
\begin{bmatrix}
r_1\\
r_2\\
r_3\\
r_0\\
\end{bmatrix}
\end{align*}


\begin{minted}{cpp}
__m128 rb03 = _mm_fmsub(r, bP1230, _mm_mul(rP1230, b));
__m128 ar03 = _mm_fmsub(a, rP1230, _mm_mul(aP1230, r));
__m128 rd03 = _mm_fmsub(r, dP1230, _mm_mul(rP1230, d));
__m128 cr03 = _mm_fmsub(c, rP1230, _mm_mul(cP1230, r));
\end{minted}

\begin{align*}
\mathrm{rb03} 
=
\begin{bmatrix}
r_0b_1 - r_1b_0\\
r_1b_2 - r_2b_1\\
r_2b_3 - r_3b_2\\
r_3b_0 - r_0b_3\\
\end{bmatrix}
&&
\mathrm{ar03} 
&=
\begin{bmatrix}
a_0r_1 - a_1r_0\\
a_1r_2 - a_2r_1\\
a_2r_3 - a_3r_2\\
a_3r_0 - a_0r_3\\
\end{bmatrix}
\\
\\
\mathrm{rd03} 
=
\begin{bmatrix}
r_0d_1 - r_1d_0\\
r_1d_2 - r_2d_1\\
r_2d_3 - r_3d_2\\
r_3d_0 - r_0d_3\\
\end{bmatrix}
&&
\mathrm{cr03} 
&=
\begin{bmatrix}
c_0r_1 - c_1r_0\\
c_1r_2 - c_2r_1\\
c_2r_3 - c_3r_2\\
c_3r_0 - c_0r_3\\
\end{bmatrix}
\end{align*}

\begin{minted}{cpp}
__m128 rd03P2301 = Permute<2, 3, 0, 1>(rd03);
__m128 cr03P2301 = Permute<2, 3, 0, 1>(cr03);
\end{minted}

\begin{align*}
\mathrm{rd03P2301} 
=
\begin{bmatrix}
r_2d_3 - r_3d_2\\
r_3d_0 - r_0d_3\\
r_0d_1 - r_1d_0\\
r_1d_2 - r_2d_1\\
\end{bmatrix}
&&
\mathrm{cr03P2301} 
&=
\begin{bmatrix}
c_2r_3 - c_3r_2\\
c_3r_0 - c_0r_3\\
c_0r_1 - c_1r_0\\
c_1r_2 - c_2r_1\\
\end{bmatrix}
\end{align*}

\begin{minted}{cpp}
__m128 rd03P2301N = Negate<0, 1, 0, 1>(rd03P2301);
__m128 cr03P2301N = Negate<0, 1, 0, 1>(cr03P2301);
\end{minted}

\begin{align*}
\mathrm{rd03P2301N} 
=
\begin{bmatrix}
r_2d_3 - r_3d_2\\
r_0d_3 - r_3d_0\\
r_0d_1 - r_1d_0\\
r_2d_1 - r_1d_2\\
\end{bmatrix}
&&
\mathrm{cr03P2301N} 
&=
\begin{bmatrix}
c_2r_3 - c_3r_2\\
c_0r_3 - c_3r_0\\
c_0r_1 - c_1r_0\\
c_2r_1 - c_1r_2\\
\end{bmatrix}
\end{align*}

\begin{minted}{cpp}
__m128 products03M0 = _mm_mul(rb03, cd03P2301N);
__m128 products03M1 = _mm_mul(ar03, cd03P2301N);
__m128 products03M2 = _mm_mul(ab03, rd03P2301N);
__m128 products03M3 = _mm_mul(ab03, cr03P2301N);
\end{minted}

\begin{align*}
\mathrm{products03M0} 
=
\begin{bmatrix}
\pth{r_0b_1 - r_1b_0}\pth{c_2d_3 - c_3d_2}\\
\pth{r_1b_2 - r_2b_1}\pth{c_3d_0 - c_0d_3}\\
\pth{r_2b_3 - r_3b_2}\pth{c_0d_1 - c_1d_0}\\
\pth{r_3b_0 - r_0b_3}\pth{c_1d_2 - c_2d_1}\\
\end{bmatrix}
&&
\mathrm{products03M1} 
&=
\begin{bmatrix}
\pth{a_0r_1 - a_1r_0}\pth{c_2d_3 - c_3d_2}\\
\pth{a_1r_2 - a_2r_1}\pth{c_3d_0 - c_0d_3}\\
\pth{a_2r_3 - a_3r_2}\pth{c_0d_1 - c_1d_0}\\
\pth{a_3r_0 - a_0r_3}\pth{c_1d_2 - c_2d_1}\\
\end{bmatrix}
\\
\mathrm{products03M2} 
=
\begin{bmatrix}
\pth{a_0b_1 - a_1b_0}\pth{r_2d_3 - r_3d_2}\\
\pth{a_1b_2 - a_2b_1}\pth{r_0d_3 - r_3d_0}\\
\pth{a_2b_3 - a_3b_2}\pth{r_0d_1 - r_1d_0}\\
\pth{a_3b_0 - a_0b_3}\pth{r_2d_1 - r_1d_2}\\
\end{bmatrix}
&&
\mathrm{products03M3} 
&=
\begin{bmatrix}
\pth{a_0b_1 - a_1b_0}\pth{c_2r_3 - c_3r_2}\\
\pth{a_1b_2 - a_2b_1}\pth{c_0r_3 - c_3r_0}\\
\pth{a_2b_3 - a_3b_2}\pth{c_0r_1 - c_1r_0}\\
\pth{a_3b_0 - a_0b_3}\pth{c_2r_1 - c_1r_2}\\
\end{bmatrix}
\end{align*}



Subsequently, the last 2 terms of the modified determinants are calculated.
\begin{minted}{cpp}
__m128 rP2301 = Permute<2, 3, 0, 1>(r);
\end{minted}

\begin{align*}
\mathrm{rP2301} 
=
\begin{bmatrix}
r_2\\
r_3\\
r_0\\
r_1\\
\end{bmatrix}
\end{align*}

\begin{minted}{cpp}
__m128 rbrd45 = _mm_fmsub(r, bdB1100P2301, _mm_mul(rP2301, bdB0011));
__m128 arcr45 = _mm_fmsub(acB0011, rP2301, _mm_mul(acB1100P2301, r));
\end{minted}

\begin{align*}
\mathrm{rbrd45} 
=
\begin{bmatrix}
r_0b_2- r_2b_0\\
r_1b_3- r_3b_1\\
r_2d_0- r_0d_2\\
r_3d_1- r_1d_3\\
\end{bmatrix}
&&
\mathrm{arcr45} 
&=
\begin{bmatrix}
a_0r_2 - a_2r_0\\
a_1r_3 - a_3r_1\\
c_2r_0 - c_0r_2\\
c_3r_1 - c_1r_3\\
\end{bmatrix}
\end{align*}

\begin{minted}{cpp}
__m128 products03M0M2B0011 = Blend<0, 0, 1, 1>(products03M0, products03M2);
__m128 products03M0M2B1100 = Blend<1, 1, 0, 0>(products03M0, products03M2);
__m128 products03M1M3B0011 = Blend<0, 0, 1, 1>(products03M1, products03M3);
__m128 products03M1M3B1100 = Blend<1, 1, 0, 0>(products03M1, products03M3);
\end{minted}

\begin{align*}
\mathrm{products03M0M2B0011} 
&=
\begin{bmatrix}
\pth{r_0b_1 - r_1b_0}\pth{c_2d_3 - c_3d_2}\\
\pth{r_1b_2 - r_2b_1}\pth{c_3d_0 - c_0d_3}\\
\pth{a_2b_3 - a_3b_2}\pth{r_0d_1 - r_1d_0}\\
\pth{a_3b_0 - a_0b_3}\pth{r_2d_1 - r_1d_2}\\
\end{bmatrix}
\\
\mathrm{products03M0M2B1100} 
&=
\begin{bmatrix}
\pth{a_0b_1 - a_1b_0}\pth{r_2d_3 - r_3d_2}\\
\pth{a_1b_2 - a_2b_1}\pth{r_0d_3 - r_3d_0}\\
\pth{r_2b_3 - r_3b_2}\pth{c_0d_1 - c_1d_0}\\
\pth{r_3b_0 - r_0b_3}\pth{c_1d_2 - c_2d_1}\\
\end{bmatrix}
\\
\mathrm{products03M1M3B0011} 
&=
\begin{bmatrix}
\pth{a_0r_1 - a_1r_0}\pth{c_2d_3 - c_3d_2}\\
\pth{a_1r_2 - a_2r_1}\pth{c_3d_0 - c_0d_3}\\
\pth{a_2b_3 - a_3b_2}\pth{c_0r_1 - c_1r_0}\\
\pth{a_3b_0 - a_0b_3}\pth{c_2r_1 - c_1r_2}\\
\end{bmatrix}
\\
\mathrm{products03M1M3B1100} 
&=
\begin{bmatrix}
\pth{a_0b_1 - a_1b_0}\pth{c_2r_3 - c_3r_2}\\
\pth{a_1b_2 - a_2b_1}\pth{c_0r_3 - c_3r_0}\\
\pth{a_2r_3 - a_3r_2}\pth{c_0d_1 - c_1d_0}\\
\pth{a_3r_0 - a_0r_3}\pth{c_1d_2 - c_2d_1}\\
\end{bmatrix}
\end{align*}


\begin{minted}{cpp}
__m128 products03M0M2B1100P2301 = Permute<2, 3, 0, 1>(products03M0M2B1100);
__m128 products03M1M3B1100P2301 = Permute<2, 3, 0, 1>(products03M1M3B1100);
\end{minted}

\begin{align*}
\mathrm{products03M0M2B1100P2301} 
=
\begin{bmatrix}
\pth{r_2b_3 - r_3b_2}\pth{c_0d_1 - c_1d_0}\\
\pth{r_3b_0 - r_0b_3}\pth{c_1d_2 - c_2d_1}\\
\pth{a_0b_1 - a_1b_0}\pth{r_2d_3 - r_3d_2}\\
\pth{a_1b_2 - a_2b_1}\pth{r_0d_3 - r_3d_0}\\
\end{bmatrix}
\\
\mathrm{products03M1M3B1100P2301} 
=
\begin{bmatrix}
\pth{a_2r_3 - a_3r_2}\pth{c_0d_1 - c_1d_0}\\
\pth{a_3r_0 - a_0r_3}\pth{c_1d_2 - c_2d_1}\\
\pth{a_0b_1 - a_1b_0}\pth{c_2r_3 - c_3r_2}\\
\pth{a_1b_2 - a_2b_1}\pth{c_0r_3 - c_3r_0}\\
\end{bmatrix}
\end{align*}

\begin{minted}{cpp}
__m128 sumsM0M2 = _mm_fmadd(rbrd45, abcd45P3210, _mm_add(products03M0M2B0011, 
                                                         products03M0M2B1100P2301));
__m128 sumsM1M3 = _mm_fmadd(arcr45, abcd45P3210, _mm_add(products03M1M3B0011, 
                                                         products03M1M3B1100P2301));    
\end{minted}

\begin{align*}
\mathrm{sumsM0M2} 
=
\begin{bmatrix}
  \pth{r_0b_2 - r_2b_0}\pth{c_3d_1 - c_1d_3}
+ \pth{r_0b_1 - r_1b_0}\pth{c_2d_3 - c_3d_2}
+ \pth{r_2b_3 - r_3b_2}\pth{c_0d_1 - c_1d_0}\\
  \pth{r_1b_3 - r_3b_1}\pth{c_2d_0 - c_0d_2}
+ \pth{r_1b_2 - r_2b_1}\pth{c_3d_0 - c_0d_3}
+ \pth{r_3b_0 - r_0b_3}\pth{c_1d_2 - c_2d_1}\\
  \pth{a_1b_3 - a_3b_1}\pth{r_2d_0 - r_0d_2}
+ \pth{a_2b_3 - a_3b_2}\pth{r_0d_1 - r_1d_0}
+ \pth{a_0b_1 - a_1b_0}\pth{r_2d_3 - r_3d_2}\\
  \pth{a_0b_2 - a_2b_0}\pth{r_3d_1 - r_1d_3}
+ \pth{a_3b_0 - a_0b_3}\pth{r_2d_1 - r_1d_2}
+ \pth{a_1b_2 - a_2b_1}\pth{r_0d_3 - r_3d_0}\\
\end{bmatrix}
\\
\mathrm{sumsM1M3} 
=
\begin{bmatrix}
  \pth{a_0r_2 - a_2r_0}\pth{c_3d_1 - c_1d_3}
+ \pth{a_0r_1 - a_1r_0}\pth{c_2d_3 - c_3d_2}
+ \pth{a_2r_3 - a_3r_2}\pth{c_0d_1 - c_1d_0}\\
  \pth{a_1r_3 - a_3r_1}\pth{c_2d_0 - c_0d_2}
+ \pth{a_1r_2 - a_2r_1}\pth{c_3d_0 - c_0d_3}
+ \pth{a_3r_0 - a_0r_3}\pth{c_1d_2 - c_2d_1}\\
  \pth{a_1b_3 - a_3b_1}\pth{c_2r_0 - c_0r_2}
+ \pth{a_2b_3 - a_3b_2}\pth{c_0r_1 - c_1r_0}
+ \pth{a_0b_1 - a_1b_0}\pth{c_2r_3 - c_3r_2}\\
  \pth{a_0b_2 - a_2b_0}\pth{c_3r_1 - c_1r_3}
+ \pth{a_3b_0 - a_0b_3}\pth{c_2r_1 - c_1r_2}
+ \pth{a_1b_2 - a_2b_1}\pth{c_0r_3 - c_3r_0}\\
\end{bmatrix}
\end{align*}

\begin{minted}{cpp}
__m128 sumsB0101 = Blend<0, 1, 0, 1>(sumsM0M2, sumsM1M3);
__m128 sumsB1010 = Blend<1, 0, 1, 0>(sumsM0M2, sumsM1M3);
\end{minted}

\begin{align*}
\mathrm{sumsB0101} 
=
\begin{bmatrix}
  \pth{r_0b_2 - r_2b_0}\pth{c_3d_1 - c_1d_3}
+ \pth{r_0b_1 - r_1b_0}\pth{c_2d_3 - c_3d_2}
+ \pth{r_2b_3 - r_3b_2}\pth{c_0d_1 - c_1d_0}\\
  \pth{a_1r_3 - a_3r_1}\pth{c_2d_0 - c_0d_2}
+ \pth{a_1r_2 - a_2r_1}\pth{c_3d_0 - c_0d_3}
+ \pth{a_3r_0 - a_0r_3}\pth{c_1d_2 - c_2d_1}\\
  \pth{a_1b_3 - a_3b_1}\pth{r_2d_0 - r_0d_2}
+ \pth{a_2b_3 - a_3b_2}\pth{r_0d_1 - r_1d_0}
+ \pth{a_0b_1 - a_1b_0}\pth{r_2d_3 - r_3d_2}\\
  \pth{a_0b_2 - a_2b_0}\pth{c_3r_1 - c_1r_3}
+ \pth{a_3b_0 - a_0b_3}\pth{c_2r_1 - c_1r_2}
+ \pth{a_1b_2 - a_2b_1}\pth{c_0r_3 - c_3r_0}\\
\end{bmatrix}
\\
\mathrm{sumsB1010} 
=
\begin{bmatrix}
  \pth{a_0r_2 - a_2r_0}\pth{c_3d_1 - c_1d_3}
+ \pth{a_0r_1 - a_1r_0}\pth{c_2d_3 - c_3d_2}
+ \pth{a_2r_3 - a_3r_2}\pth{c_0d_1 - c_1d_0}\\
  \pth{r_1b_3 - r_3b_1}\pth{c_2d_0 - c_0d_2}
+ \pth{r_1b_2 - r_2b_1}\pth{c_3d_0 - c_0d_3}
+ \pth{r_3b_0 - r_0b_3}\pth{c_1d_2 - c_2d_1}\\
  \pth{a_1b_3 - a_3b_1}\pth{c_2r_0 - c_0r_2}
+ \pth{a_2b_3 - a_3b_2}\pth{c_0r_1 - c_1r_0}
+ \pth{a_0b_1 - a_1b_0}\pth{c_2r_3 - c_3r_2}\\
  \pth{a_0b_2 - a_2b_0}\pth{r_3d_1 - r_1d_3}
+ \pth{a_3b_0 - a_0b_3}\pth{r_2d_1 - r_1d_2}
+ \pth{a_1b_2 - a_2b_1}\pth{r_0d_3 - r_3d_0}\\
\end{bmatrix}
\end{align*}


\begin{minted}{cpp}
__m128 sumsB1010P1032 = Permute<1, 0, 3, 2>(sumsB1010);
\end{minted}

\begin{align*}
&\mathrm{sumsB1010P1032} 
=\\
&\begin{bmatrix}
  \pth{r_1b_3 - r_3b_1}\pth{c_2d_0 - c_0d_2}
+ \pth{r_1b_2 - r_2b_1}\pth{c_3d_0 - c_0d_3}
+ \pth{r_3b_0 - r_0b_3}\pth{c_1d_2 - c_2d_1}\\
  \pth{a_0r_2 - a_2r_0}\pth{c_3d_1 - c_1d_3}
+ \pth{a_0r_1 - a_1r_0}\pth{c_2d_3 - c_3d_2}
+ \pth{a_2r_3 - a_3r_2}\pth{c_0d_1 - c_1d_0}\\
  \pth{a_0b_2 - a_2b_0}\pth{r_3d_1 - r_1d_3}
+ \pth{a_3b_0 - a_0b_3}\pth{r_2d_1 - r_1d_2}
+ \pth{a_1b_2 - a_2b_1}\pth{r_0d_3 - r_3d_0}\\
  \pth{a_1b_3 - a_3b_1}\pth{c_2r_0 - c_0r_2}
+ \pth{a_2b_3 - a_3b_2}\pth{c_0r_1 - c_1r_0}
+ \pth{a_0b_1 - a_1b_0}\pth{c_2r_3 - c_3r_2}\\
\end{bmatrix}
\end{align*}

\begin{minted}{cpp}
__m128 determinants = _mm_add(sumsB0101, sumsB1010P1032);
\end{minted}


\begin{align*}
\mathrm{determinants} 
=
\begin{bmatrix}
\mathrm{det}\pth{\mathbf{M}_0}\\
\mathrm{det}\pth{\mathbf{M}_1}\\
\mathrm{det}\pth{\mathbf{M}_2}\\
\mathrm{det}\pth{\mathbf{M}_3}\\
\end{bmatrix}
\end{align*}

Finally, the system's solution is calculated and returned.

\begin{minted}{cpp}
__m128 solution = _mm_div(determinants, detA);

return Vec4fSSE<true>(solution);
\end{minted}


\vspace{1cm}
\textbf{Complete Function:}

\begin{minted}{cpp}
inline Vec4fSSE<true> Cramer(const Mat4fSSE& matA, const Vec4fSSE<true>& vecRhs)
{
    using namespace GDL::simd;

    const std::array<__m128, 4>& matrixData = matA.DataSSE();
    const __m128& a = matrixData[0];
    const __m128& b = matrixData[1];
    const __m128& c = matrixData[2];
    const __m128& d = matrixData[3];


    // Calculate sum of first 4 terms of det(A)
    __m128 aP1230 = Permute<1, 2, 3, 0>(a);
    __m128 bP1230 = Permute<1, 2, 3, 0>(b);
    __m128 cP1230 = Permute<1, 2, 3, 0>(c);
    __m128 dP1230 = Permute<1, 2, 3, 0>(d);

    __m128 ab03 = _mm_fmsub(a, bP1230, _mm_mul(aP1230, b));
    __m128 cd03 = _mm_fmsub(c, dP1230, _mm_mul(cP1230, d));

    __m128 cd03P2301 = Permute<2, 3, 0, 1>(cd03);
    __m128 cd03P2301N = Negate<0, 1, 0, 1>(cd03P2301);

    __m128 sum03 = DotProduct(ab03, cd03P2301N);


    // Calculate sum of last 2 terms of det(A)
    __m128 acB0011 = Blend<0, 0, 1, 1>(a, c);
    __m128 acB1100 = Blend<1, 1, 0, 0>(a, c);
    __m128 bdB0011 = Blend<0, 0, 1, 1>(b, d);
    __m128 bdB1100 = Blend<1, 1, 0, 0>(b, d);

    __m128 acB1100P2301 = Permute<2, 3, 0, 1>(acB1100);
    __m128 bdB1100P2301 = Permute<2, 3, 0, 1>(bdB1100);

    __m128 abcd45 = _mm_fmsub(acB0011, bdB1100P2301, _mm_mul(acB1100P2301, bdB0011));
    __m128 abcd45P3210 = Permute<3, 2, 1, 0>(abcd45);

    __m128 products45 = _mm_mul(abcd45, abcd45P3210);
    __m128 sum45 = _mm_add(products45, Permute<1, 0, 3, 2>(products45));


    // Calculate det(A)
    __m128 detA = _mm_add(sum03, sum45);


    DEV_EXCEPTION(_mm_cvtsF(detA) == ApproxZero<F32>(10), 
                  "Singular matrix - system not solveable");


    // Calculate the first 4 terms of the modified matrix determinants
    const __m128& r = vecRhs.DataSSE();
    __m128 rP1230 = Permute<1, 2, 3, 0>(r);

    __m128 rb03 = _mm_fmsub(r, bP1230, _mm_mul(rP1230, b));
    __m128 ar03 = _mm_fmsub(a, rP1230, _mm_mul(aP1230, r));
    __m128 rd03 = _mm_fmsub(r, dP1230, _mm_mul(rP1230, d));
    __m128 cr03 = _mm_fmsub(c, rP1230, _mm_mul(cP1230, r));

    __m128 rd03P2301 = Permute<2, 3, 0, 1>(rd03);
    __m128 cr03P2301 = Permute<2, 3, 0, 1>(cr03);

    __m128 rd03P2301N = Negate<0, 1, 0, 1>(rd03P2301);
    __m128 cr03P2301N = Negate<0, 1, 0, 1>(cr03P2301);

    __m128 products03M0 = _mm_mul(rb03, cd03P2301N);
    __m128 products03M1 = _mm_mul(ar03, cd03P2301N);
    __m128 products03M2 = _mm_mul(ab03, rd03P2301N);
    __m128 products03M3 = _mm_mul(ab03, cr03P2301N);


    // Calculate last 2 terms of the modified determinants
    __m128 rP2301 = Permute<2, 3, 0, 1>(r);

    __m128 rbrd45 = _mm_fmsub(r, bdB1100P2301, _mm_mul(rP2301, bdB0011));
    __m128 arcr45 = _mm_fmsub(acB0011, rP2301, _mm_mul(acB1100P2301, r));


    // Calculate modified determinants
    __m128 products03M0M2B0011 = Blend<0, 0, 1, 1>(products03M0, products03M2);
    __m128 products03M0M2B1100 = Blend<1, 1, 0, 0>(products03M0, products03M2);
    __m128 products03M1M3B0011 = Blend<0, 0, 1, 1>(products03M1, products03M3);
    __m128 products03M1M3B1100 = Blend<1, 1, 0, 0>(products03M1, products03M3);

    __m128 products03M0M2B1100P2301 = Permute<2, 3, 0, 1>(products03M0M2B1100);
    __m128 products03M1M3B1100P2301 = Permute<2, 3, 0, 1>(products03M1M3B1100);

    __m128 sumsM0M2 
        = _mm_fmadd(rbrd45, abcd45P3210, _mm_add(products03M0M2B0011, 
                                                 products03M0M2B1100P2301));
    __m128 sumsM1M3 
        = _mm_fmadd(arcr45, abcd45P3210, _mm_add(products03M1M3B0011, 
                                                 products03M1M3B1100P2301));

    __m128 sumsB0101 = Blend<0, 1, 0, 1>(sumsM0M2, sumsM1M3);
    __m128 sumsB1010 = Blend<1, 0, 1, 0>(sumsM0M2, sumsM1M3);

    __m128 sumsB1010P1032 = Permute<1, 0, 3, 2>(sumsB1010);

    __m128 determinants = _mm_add(sumsB0101, sumsB1010P1032);


    // Calculate and return solution
    __m128 solution = _mm_div(determinants, detA);

    return Vec4fSSE<true>(solution);
}
\end{minted}




% ------------------------------------------------------------------------------
\newpage
\subsubsection{4x4 determinant - AVX}
\label{sec:determinant4x4AVX}

The matrix 

\begin{align*}
\mathbf{A}
=
\begin{bmatrix}
a_0&b_0&c_0&d_0\\
a_1&b_1&c_1&d_1\\
a_2&b_2&c_2&d_2\\
a_3&b_3&c_3&d_3
\end{bmatrix}
\end{align*}

is stored in two \mintinline{cpp}{__m256} registers as follows:

\begin{align*}
\mathrm{ab} 
=
\begin{bmatrix}
a_0\\
a_1\\
a_2\\
a_3\\
b_0\\
b_1\\
b_2\\
b_3\\
\end{bmatrix}
&&
\mathrm{cd} 
&=
\begin{bmatrix}
c_0\\
c_1\\
c_2\\
c_3\\
d_0\\
d_1\\
d_2\\
d_3\\
\end{bmatrix}
\end{align*}

The AVX algorithm to calculate the determinant is based on \cref{eq:determinant4x4AVX}.
Since it only needs to be translated into a minimal set of AVX instructions and no complicated operations are involved, there won't be a detailed explanation of each line of code.
Instead, the content of the affected variables is shown after each code segment.

The algorithm starts by calculating the components of the first four terms of \cref{eq:determinant4x4AVX}.

\begin{minted}{cpp}
__m256 abP1230 = Permute<1, 2, 3, 0>(ab);
__m256 cdP1230 = Permute<1, 2, 3, 0>(cd);
\end{minted}
\begin{align*}
\mathrm{abP1230} 
=
\begin{bmatrix}
a_1\\
a_2\\
a_3\\
a_0\\
b_1\\
b_2\\
b_3\\
b_0\\
\end{bmatrix}
&&
\mathrm{cdP1230} 
=
\begin{bmatrix}
c_1\\
c_2\\
c_3\\
c_0\\
d_1\\
d_2\\
d_3\\
d_0\\
\end{bmatrix}
\end{align*}



\begin{minted}{cpp}
__m256 acbd03 = _mm_fmsub(ab, cdP1230, _mm_mul(cd, abP1230)); 
\end{minted}
\begin{align*}
\mathrm{acbd03} 
&=
\begin{bmatrix}
a_0c_1 - a_1c_0\\
a_1c_2 - a_2c_1\\
a_2c_3 - a_3c_2\\
a_3c_0 - a_0c_3\\
b_0d_1 - b_1d_0\\
b_1d_2 - b_2d_1\\
b_2d_3 - b_3d_2\\
b_3d_0 - b_0d_3\\
\end{bmatrix}
\end{align*}



\begin{minted}{cpp}
__m256 acbd03N = Negate<0, 0, 0, 0, 1, 0, 1, 0>(acbd03);
\end{minted}
\begin{align*}
\mathrm{acbd03N} 
&=
\begin{bmatrix}
a_0c_1 - a_1c_0\\
a_1c_2 - a_2c_1\\
a_2c_3 - a_3c_2\\
a_3c_0 - a_0c_3\\
b_1d_0 - b_0d_1\\
b_1d_2 - b_2d_1\\
b_3d_2 - b_2d_3\\
b_3d_0 - b_0d_3\\
\end{bmatrix}
\end{align*}



Subsequently, the components of the last two terms are calculated .

\begin{minted}{cpp}
__m256 abP2323 = Permute<2, 3, 2, 3>(ab);
__m256 cdP2323 = Permute<2, 3, 2, 3>(cd);
\end{minted}
\begin{align*}
\mathrm{abP2323} 
=
\begin{bmatrix}
a_2\\
a_3\\
a_2\\
a_3\\
b_2\\
b_3\\
b_2\\
b_3\\
\end{bmatrix}
&&
\mathrm{cdP2323} 
&=
\begin{bmatrix}
c_2\\
c_3\\
c_2\\
c_3\\
d_2\\
d_3\\
d_2\\
d_3\\
\end{bmatrix}
\end{align*}


\begin{minted}{cpp}
__m256 acbd45 = _mm_fmsub(ab, cdP2323, _mm_mul(cd, abP2323));
\end{minted}
\begin{align*}
\mathrm{acbd45} 
&=
\begin{bmatrix}
a_0c_2 - a_2c_0\\
a_1c_3 - a_3c_1\\
a_2c_2 - a_2c_2\\
a_3c_3 - a_3c_3\\
b_0d_2 - b_2d_0\\
b_1d_3 - b_3d_1\\
b_2d_2 - b_2d_2\\
b_3d_3 - b_3d_3\\
\end{bmatrix}
=
\begin{bmatrix}
a_0c_2 - a_2c_0\\
a_1c_3 - a_3c_1\\
0\\
0\\
b_0d_2 - b_2d_0\\
b_1d_3 - b_3d_1\\
0\\
0\\
\end{bmatrix}
\end{align*}

Now the determinant can be determined using the previously calculated values.

\begin{minted}{cpp}
__m256 ac03bd45 = Blend<0, 0, 0, 0, 1, 1, 1, 1>(acbd03N, acbd45);
\end{minted}
\begin{align*}
\mathrm{ac03bd45} 
&=
\begin{bmatrix}
a_0c_1 - a_1c_0\\
a_1c_2 - a_2c_1\\
a_2c_3 - a_3c_2\\
a_3c_0 - a_0c_3\\
b_0d_2 - b_2d_0\\
b_1d_3 - b_3d_1\\
0\\
0\\
\end{bmatrix}
\end{align*}

\begin{minted}{cpp}
__m256 bd03ac45 = Permute2F128<0, 1, 1, 0>(acbd03N, acbd45);
\end{minted}
\begin{align*}
\mathrm{bd03ac45} 
&=
\begin{bmatrix}
b_1d_0 - b_0d_1\\
b_1d_2 - b_2d_1\\
b_3d_2 - b_2d_3\\
b_3d_0 - b_0d_3\\
a_0c_2 - a_2c_0\\
a_1c_3 - a_3c_1\\
0\\
0\\
\end{bmatrix}
\end{align*}

\begin{minted}{cpp}
__m256 bd03ac45P = Permute<2, 3, 0, 1, 1, 0, 2, 3>(bd03ac45);
\end{minted}
\begin{align*}
\mathrm{bd03ac45P} 
&=
\begin{bmatrix}
b_3d_2 - b_2d_3\\
b_3d_0 - b_0d_3\\
b_1d_0 - b_0d_1\\
b_1d_2 - b_2d_1\\
a_1c_3 - a_3c_1\\
a_0c_2 - a_2c_0\\
0\\
0\\
\end{bmatrix}
\end{align*}

\begin{minted}{cpp}
__m256 sums = DotProduct(ac03bd45, bd03ac45P);
\end{minted}
\begin{align*}
\mathrm{sums} 
&=
\begin{bmatrix}
\mathrm{sum03}\\
\mathrm{sum03}\\
\mathrm{sum03}\\
\mathrm{sum03}\\
\mathrm{sum45}\\
\mathrm{sum45}\\
\mathrm{sum45}\\
\mathrm{sum45}\\
\end{bmatrix}\\
\\
\mathrm{sum03}
&=\pth{a_0c_1 - a_1c_0}\pth{b_3d_2 - b_2d_3}\\
&+\pth{a_1c_2 - a_2c_1}\pth{b_3d_0 - b_0d_3}\\
&+\pth{a_2c_3 - a_3c_2}\pth{b_1d_0 - b_0d_1}\\
&+\pth{a_3c_0 - a_0c_3}\pth{b_1d_2 - b_2d_1}\\
\\
\mathrm{sum45}
&=\pth{a_1c_3 - a_3c_1}\pth{b_0d_2 - b_2d_0}\\
&+\pth{a_0c_2 - a_2c_0}\pth{b_1d_3 - b_3d_1}
\end{align*}

Finally, the determinant is calculated and returned:
\begin{minted}{cpp}
__m256 determinant = _mm_add(sums, Permute2F128<1, 0>(sums));

return _mm_cvtsF(determinant);
\end{minted}



\vspace{1cm}
\textbf{Complete Function:}
\begin{minted}{cpp}
inline F32 Determinant4x4(__m256 ab, __m256 cd)
{
    // Calculate the components of the first 4 terms
    __m256 abP1230 = Permute<1, 2, 3, 0>(ab);
    __m256 cdP1230 = Permute<1, 2, 3, 0>(cd);

    __m256 acbd03 = _mm_fmsub(ab, cdP1230, _mm_mul(cd, abP1230));
    __m256 acbd03N = Negate<0, 0, 0, 0, 1, 0, 1, 0>(acbd03);


    // Calculate the components of the last 2 terms
    __m256 abP2323 = Permute<2, 3, 2, 3>(ab);
    __m256 cdP2323 = Permute<2, 3, 2, 3>(cd);
    __m256 acbd45 = _mm_fmsub(ab, cdP2323, _mm_mul(cd, abP2323));


    // Calculate determinant
    __m256 ac03bd45 = Blend<0, 0, 0, 0, 1, 1, 1, 1>(acbd03N, acbd45);
    __m256 bd03ac45 = Permute2F128<0, 1, 1, 0>(acbd03N, acbd45);
    __m256 bd03ac45P = Permute<2, 3, 0, 1, 1, 0, 2, 3>(bd03ac45);

    __m256 sums = DotProduct(ac03bd45, bd03ac45P);

    __m256 determinant = _mm_add(sums, Permute2F128<1, 0>(sums));

    return _mm_cvtsF(determinant);
}
\end{minted}




% ------------------------------------------------------------------------------
\newpage
\subsubsection{4x4 solver - AVX}

First the determinant of $\mathbf{A}$ is calculated.
Because some intermediate results can be reused, the previously introduced determinant function of \cref{sec:determinant4x4AVX} is not employed here. 
Instead, its code is copied. 
The resulting code duplication can be justified by the huge performance gain.

The variable \mintinline{cpp}{determinant} of \cref{sec:determinant4x4AVX} is renamed to \mintinline{cpp}{detA}.

After the calculation of $\mathrm{det}\mathbf{A}$ it is used to test if the system is solvable.

\begin{minted}{cpp}
DEV_EXCEPTION(_mm_cvtsF(detA) == ApproxZero<F32>(10), 
              "Singular matrix - system not solveable");
\end{minted} 

The values of the right-hand side vector are stored in both lanes of a $\mintinline{cpp}{__m256}$ register:


\begin{align*}
\mathrm{r} 
=
\begin{bmatrix}
r_0\\
r_1\\
r_2\\
r_3\\
r_0\\
r_1\\
r_2\\
r_3
\end{bmatrix}
\end{align*}


Then the missing values to calculate the first 4 terms of \cref{eq:determinant4x4AVX} for the modified determinants are obtained by the following instructions:

\begin{minted}{cpp}
__m256 rP1230 = Permute<1, 2, 3, 0>(r);
\end{minted} 

\begin{align*}
\mathrm{rP1230} 
=
\begin{bmatrix}
r_1\\
r_2\\
r_3\\
r_0\\
r_1\\
r_2\\
r_3\\
r_0\\
\end{bmatrix}
\end{align*}

\begin{minted}{cpp}
__m256 rcrd03 = _mm_fmsub(r, cdP1230, _mm_mul(rP1230, cd));
__m256 arbr03 = _mm_fmsub(ab, rP1230, _mm_mul(r, abP1230));
\end{minted} 

\begin{align*}
\mathrm{rcrd03} 
=
\begin{bmatrix}
r_0c_1 - r_1c_0\\
r_1c_2 - r_2c_1\\
r_2c_3 - r_3c_2\\
r_3c_0 - r_0c_3\\
r_0d_1 - r_1d_0\\
r_1d_2 - r_2d_1\\
r_2d_3 - r_3d_2\\
r_3d_0 - r_0d_3\\
\end{bmatrix}
&&
\mathrm{arbr03} 
&=
\begin{bmatrix}
a_0r_1 - a_1r_0\\
a_1r_2 - a_2r_1\\
a_2r_3 - a_3r_2\\
a_3r_0 - a_0r_3\\
b_0r_1 - b_1r_0\\
b_1r_2 - b_2r_1\\
b_2r_3 - b_3r_2\\
b_3r_0 - b_0r_3\\
\end{bmatrix}
\end{align*}

\begin{minted}{cpp}
__m256 rcrd03N = Negate<0, 0, 0, 0, 1, 0, 1, 0>(rcrd03);
__m256 arbr03N = Negate<0, 0, 0, 0, 1, 0, 1, 0>(arbr03);
\end{minted} 

\begin{align*}
\mathrm{rcrd03N} 
=
\begin{bmatrix}
r_0c_1 - r_1c_0\\
r_1c_2 - r_2c_1\\
r_2c_3 - r_3c_2\\
r_3c_0 - r_0c_3\\
r_1d_0 - r_0d_1\\
r_1d_2 - r_2d_1\\
r_3d_2 - r_2d_3\\
r_3d_0 - r_0d_3\\
\end{bmatrix}
&&
\mathrm{arbr03N} 
&=
\begin{bmatrix}
a_0r_1 - a_1r_0\\
a_1r_2 - a_2r_1\\
a_2r_3 - a_3r_2\\
a_3r_0 - a_0r_3\\
b_1r_0 - b_0r_1\\
b_1r_2 - b_2r_1\\
b_3r_2 - b_2r_3\\
b_3r_0 - b_0r_3\\
\end{bmatrix}
\end{align*}

\begin{minted}{cpp}
__m256 bdac03 = Permute2F128<1, 0>(acbd03N);
\end{minted} 

\begin{align*}
\mathrm{bdac03} 
&=
\begin{bmatrix}
b_1d_0 - b_0d_1\\
b_1d_2 - b_2d_1\\
b_3d_2 - b_2d_3\\
b_3d_0 - b_0d_3\\
a_0c_1 - a_1c_0\\
a_1c_2 - a_2c_1\\
a_2c_3 - a_3c_2\\
a_3c_0 - a_0c_3\\
\end{bmatrix}
\end{align*}

\begin{minted}{cpp}
__m256 bdac03P2301 = Permute<2, 3, 0, 1>(bdac03);
\end{minted} 
\begin{align*}
\mathrm{bdac03P2301} 
&=
\begin{bmatrix}
b_3d_2 - b_2d_3\\
b_3d_0 - b_0d_3\\
b_1d_0 - b_0d_1\\
b_1d_2 - b_2d_1\\
a_2c_3 - a_3c_2\\
a_3c_0 - a_0c_3\\
a_0c_1 - a_1c_0\\
a_1c_2 - a_2c_1\\
\end{bmatrix}
\end{align*}


\begin{minted}{cpp}
__m256 products03M0M1 = _mm_mul(rcrd03N, bdac03P2301);
__m256 products03M2M3 = _mm_mul(arbr03N, bdac03P2301);
\end{minted} 
\begin{align*}
\mathrm{products03M0M1} 
=
\begin{bmatrix}
\pth{r_0c_1 - r_1c_0}\pth{b_3d_2 - b_2d_3}\\
\pth{r_1c_2 - r_2c_1}\pth{b_3d_0 - b_0d_3}\\
\pth{r_2c_3 - r_3c_2}\pth{b_1d_0 - b_0d_1}\\
\pth{r_3c_0 - r_0c_3}\pth{b_1d_2 - b_2d_1}\\
\pth{a_2c_3 - a_3c_2}\pth{r_1d_0 - r_0d_1}\\
\pth{a_3c_0 - a_0c_3}\pth{r_1d_2 - r_2d_1}\\
\pth{a_0c_1 - a_1c_0}\pth{r_3d_2 - r_2d_3}\\
\pth{a_1c_2 - a_2c_1}\pth{r_3d_0 - r_0d_3}\\
\end{bmatrix}
&&
\mathrm{products03M2M3} 
=&
\begin{bmatrix}
\pth{a_0r_1 - a_1r_0}\pth{b_3d_2 - b_2d_3}\\
\pth{a_1r_2 - a_2r_1}\pth{b_3d_0 - b_0d_3}\\
\pth{a_2r_3 - a_3r_2}\pth{b_1d_0 - b_0d_1}\\
\pth{a_3r_0 - a_0r_3}\pth{b_1d_2 - b_2d_1}\\
\pth{a_2c_3 - a_3c_2}\pth{b_1r_0 - b_0r_1}\\
\pth{a_3c_0 - a_0c_3}\pth{b_1r_2 - b_2r_1}\\
\pth{a_0c_1 - a_1c_0}\pth{b_3r_2 - b_2r_3}\\
\pth{a_1c_2 - a_2c_1}\pth{b_3r_0 - b_0r_3}\\
\end{bmatrix}
\end{align*}

\begin{minted}{cpp}
__m256 products03B0011 
    = Blend<0, 0, 1, 1, 0, 0, 1, 1>(products03M0M1, products03M2M3);
__m256 products03B1100 
    = Blend<1, 1, 0, 0, 1, 1, 0, 0>(products03M0M1, products03M2M3);
\end{minted} 
\begin{align*}
\mathrm{products03B0011} 
=
\begin{bmatrix}
\pth{r_0c_1 - r_1c_0}\pth{b_3d_2 - b_2d_3}\\
\pth{r_1c_2 - r_2c_1}\pth{b_3d_0 - b_0d_3}\\
\pth{a_2r_3 - a_3r_2}\pth{b_1d_0 - b_0d_1}\\
\pth{a_3r_0 - a_0r_3}\pth{b_1d_2 - b_2d_1}\\
\pth{a_2c_3 - a_3c_2}\pth{r_1d_0 - r_0d_1}\\
\pth{a_3c_0 - a_0c_3}\pth{r_1d_2 - r_2d_1}\\
\pth{a_0c_1 - a_1c_0}\pth{b_3r_2 - b_2r_3}\\
\pth{a_1c_2 - a_2c_1}\pth{b_3r_0 - b_0r_3}\\
\end{bmatrix}
&&
\mathrm{products03B1100} 
&=
\begin{bmatrix}
\pth{a_0r_1 - a_1r_0}\pth{b_3d_2 - b_2d_3}\\
\pth{a_1r_2 - a_2r_1}\pth{b_3d_0 - b_0d_3}\\
\pth{r_2c_3 - r_3c_2}\pth{b_1d_0 - b_0d_1}\\
\pth{r_3c_0 - r_0c_3}\pth{b_1d_2 - b_2d_1}\\
\pth{a_2c_3 - a_3c_2}\pth{b_1r_0 - b_0r_1}\\
\pth{a_3c_0 - a_0c_3}\pth{b_1r_2 - b_2r_1}\\
\pth{a_0c_1 - a_1c_0}\pth{r_3d_2 - r_2d_3}\\
\pth{a_1c_2 - a_2c_1}\pth{r_3d_0 - r_0d_3}\\
\end{bmatrix}
\end{align*}

\begin{minted}{cpp}
__m256 products03B1100P2301 = Permute<2, 3, 0, 1>(products03B1100);
\end{minted} 
\begin{align*}
\mathrm{products03B1100P2301} 
=
\begin{bmatrix}
\pth{r_2c_3 - r_3c_2}\pth{b_1d_0 - b_0d_1}\\
\pth{r_3c_0 - r_0c_3}\pth{b_1d_2 - b_2d_1}\\
\pth{a_0r_1 - a_1r_0}\pth{b_3d_2 - b_2d_3}\\
\pth{a_1r_2 - a_2r_1}\pth{b_3d_0 - b_0d_3}\\
\pth{a_0c_1 - a_1c_0}\pth{r_3d_2 - r_2d_3}\\
\pth{a_1c_2 - a_2c_1}\pth{r_3d_0 - r_0d_3}\\
\pth{a_2c_3 - a_3c_2}\pth{b_1r_0 - b_0r_1}\\
\pth{a_3c_0 - a_0c_3}\pth{b_1r_2 - b_2r_1}\\
\end{bmatrix}
\end{align*}

\begin{minted}{cpp}
__m256 sum03 = _mm_add(products03B0011, products03B1100P2301);
\end{minted} 
\begin{align*}
\mathrm{sum03} 
=
\begin{bmatrix}
  \pth{r_0c_1 - r_1c_0}\pth{b_3d_2 - b_2d_3} 
+ \pth{r_2c_3 - r_3c_2}\pth{b_1d_0 - b_0d_1}\\
  \pth{r_1c_2 - r_2c_1}\pth{b_3d_0 - b_0d_3}
+ \pth{r_3c_0 - r_0c_3}\pth{b_1d_2 - b_2d_1}\\
  \pth{a_2r_3 - a_3r_2}\pth{b_1d_0 - b_0d_1}
+ \pth{a_0r_1 - a_1r_0}\pth{b_3d_2 - b_2d_3}\\
  \pth{a_3r_0 - a_0r_3}\pth{b_1d_2 - b_2d_1}
+ \pth{a_1r_2 - a_2r_1}\pth{b_3d_0 - b_0d_3}\\
  \pth{a_2c_3 - a_3c_2}\pth{r_1d_0 - r_0d_1}
+ \pth{a_0c_1 - a_1c_0}\pth{r_3d_2 - r_2d_3}\\
  \pth{a_3c_0 - a_0c_3}\pth{r_1d_2 - r_2d_1}
+ \pth{a_1c_2 - a_2c_1}\pth{r_3d_0 - r_0d_3}\\
  \pth{a_0c_1 - a_1c_0}\pth{b_3r_2 - b_2r_3}
+ \pth{a_2c_3 - a_3c_2}\pth{b_1r_0 - b_0r_1}\\
  \pth{a_1c_2 - a_2c_1}\pth{b_3r_0 - b_0r_3}
+ \pth{a_3c_0 - a_0c_3}\pth{b_1r_2 - b_2r_1}\\
\end{bmatrix}
\end{align*}




Subsequently, the last 2 terms of the modified determinants are calculated.
\begin{minted}{cpp}
__m256 acbdB1100 = Blend<1, 1, 0, 0, 1, 1, 0, 0>(ab, cd);
__m256 acbdB0011 = Blend<0, 0, 1, 1, 0, 0, 1, 1>(ab, cd);
\end{minted} 
\begin{align*}
\mathrm{acbdB1100} 
=
\begin{bmatrix}
a_0\\
a_1\\
c_2\\
c_3\\
b_0\\
b_1\\
d_2\\
d_3\\
\end{bmatrix}
&&
\mathrm{acbdB0011} 
=
\begin{bmatrix}
c_0\\
c_1\\
a_2\\
a_3\\
d_0\\
d_1\\
b_2\\
b_3\\
\end{bmatrix}
\end{align*}

\begin{minted}{cpp}
__m256 acbdB0011P2301 = Permute<2, 3, 0, 1>(acbdB0011);
\end{minted} 
\begin{align*}
\mathrm{acbdB0011P2301} 
=
\begin{bmatrix}
c_2\\
c_3\\
a_0\\
a_1\\
d_2\\
d_3\\
b_0\\
b_1\\
\end{bmatrix}
\end{align*}


\begin{minted}{cpp}
__m256 rP2301 = Permute<2, 3, 0, 1>(r);
\end{minted} 
\begin{align*}
\mathrm{rP2301} 
=
\begin{bmatrix}
r_2\\
r_3\\
r_0\\
r_1\\
r_2\\
r_3\\
r_0\\
r_1\\
\end{bmatrix}
\end{align*}

\begin{minted}{cpp}
__m256 components45 = _mm_fmsub(acbdB0011P2301, r, _mm_mul(rP2301, acbdB1100));
\end{minted} 
\begin{align*}
\mathrm{components45} 
=
\begin{bmatrix}
r_0c_2 - r_2c_0\\
r_1c_3 - r_3c_1\\
a_0r_2 - a_2r_0\\
a_1r_3 - a_3r_1\\
r_0d_2 - r_2d_0\\
r_1d_3 - r_3d_1\\
b_0r_2 - b_2r_0\\
b_1r_3 - b_3r_1\\
\end{bmatrix}
\end{align*}


\begin{minted}{cpp}
__m256 bdac45 = Permute2F128<1, 0>(acbd45);
\end{minted} 
\begin{align*}
\mathrm{bdac45} 
&=
\begin{bmatrix}
b_0d_2 - b_2d_0\\
b_1d_3 - b_3d_1\\
0\\
0\\
a_0c_2 - a_2c_0\\
a_1c_3 - a_3c_1\\
0\\
0\\
\end{bmatrix}
\end{align*}

\begin{minted}{cpp}
__m256 bdac45P1010 = Permute<1, 0, 1, 0>(bdac45);
\end{minted} 
\begin{align*}
\mathrm{bdac45P1010} 
&=
\begin{bmatrix}
b_1d_3 - b_3d_1\\
b_0d_2 - b_2d_0\\
b_1d_3 - b_3d_1\\
b_0d_2 - b_2d_0\\
a_1c_3 - a_3c_1\\	
a_0c_2 - a_2c_0\\
a_1c_3 - a_3c_1\\	
a_0c_2 - a_2c_0\\	
\end{bmatrix}
\end{align*}


\begin{minted}{cpp}
__m256 products45M = _mm_mul(components45, bdac45P1010);
\end{minted} 
\begin{align*}
\mathrm{products45M} 
=
\begin{bmatrix}
\pth{r_0c_2 - r_2c_0}\pth{b_1d_3 - b_3d_1}\\
\pth{r_1c_3 - r_3c_1}\pth{b_0d_2 - b_2d_0}\\
\pth{a_0r_2 - a_2r_0}\pth{b_1d_3 - b_3d_1}\\
\pth{a_1r_3 - a_3r_1}\pth{b_0d_2 - b_2d_0}\\
\pth{a_1c_3 - a_3c_1}\pth{r_0d_2 - r_2d_0}\\
\pth{a_0c_2 - a_2c_0}\pth{r_1d_3 - r_3d_1}\\
\pth{a_1c_3 - a_3c_1}\pth{b_0r_2 - b_2r_0}\\
\pth{a_0c_2 - a_2c_0}\pth{b_1r_3 - b_3r_1}\\
\end{bmatrix}
\end{align*}

Now the modified matrix determinants can be determined using the previously calculated values.

\begin{minted}{cpp}
__m256 sumsM = _mm_add(sum03, products45M);
\end{minted} 
\begin{align*}
\mathrm{sumsM} 
=
\begin{bmatrix}
  \pth{r_0c_1 - r_1c_0}\pth{b_3d_2 - b_2d_3} 
+ \pth{r_2c_3 - r_3c_2}\pth{b_1d_0 - b_0d_1}
+ \pth{r_0c_2 - r_2c_0}\pth{b_1d_3 - b_3d_1}\\
  \pth{r_1c_2 - r_2c_1}\pth{b_3d_0 - b_0d_3}
+ \pth{r_3c_0 - r_0c_3}\pth{b_1d_2 - b_2d_1}
+ \pth{r_1c_3 - r_3c_1}\pth{b_0d_2 - b_2d_0}\\
  \pth{a_2r_3 - a_3r_2}\pth{b_1d_0 - b_0d_1}
+ \pth{a_0r_1 - a_1r_0}\pth{b_3d_2 - b_2d_3}
+ \pth{a_0r_2 - a_2r_0}\pth{b_1d_3 - b_3d_1}\\
  \pth{a_3r_0 - a_0r_3}\pth{b_1d_2 - b_2d_1}
+ \pth{a_1r_2 - a_2r_1}\pth{b_3d_0 - b_0d_3}
+ \pth{a_1r_3 - a_3r_1}\pth{b_0d_2 - b_2d_0}\\
  \pth{a_2c_3 - a_3c_2}\pth{r_1d_0 - r_0d_1}
+ \pth{a_0c_1 - a_1c_0}\pth{r_3d_2 - r_2d_3}
+ \pth{a_1c_3 - a_3c_1}\pth{r_0d_2 - r_2d_0}\\
  \pth{a_3c_0 - a_0c_3}\pth{r_1d_2 - r_2d_1}
+ \pth{a_1c_2 - a_2c_1}\pth{r_3d_0 - r_0d_3}
+ \pth{a_0c_2 - a_2c_0}\pth{r_1d_3 - r_3d_1}\\
  \pth{a_0c_1 - a_1c_0}\pth{b_3r_2 - b_2r_3}
+ \pth{a_2c_3 - a_3c_2}\pth{b_1r_0 - b_0r_1}
+ \pth{a_1c_3 - a_3c_1}\pth{b_0r_2 - b_2r_0}\\
  \pth{a_1c_2 - a_2c_1}\pth{b_3r_0 - b_0r_3}
+ \pth{a_3c_0 - a_0c_3}\pth{b_1r_2 - b_2r_1}
+ \pth{a_0c_2 - a_2c_0}\pth{b_1r_3 - b_3r_1}\\
\end{bmatrix}
\end{align*}

\begin{minted}{cpp}
__m256 determinants00221133 = _mm_add(sumsM, Permute<1, 0, 3, 2>(sumsM));
\end{minted} 
\begin{align*}
\mathrm{determinants00221133} 
&=
\begin{bmatrix}
\mathrm{det}\pth{\mathbf{M}_0}\\
\mathrm{det}\pth{\mathbf{M}_0}\\
\mathrm{det}\pth{\mathbf{M}_2}\\
\mathrm{det}\pth{\mathbf{M}_2}\\
\mathrm{det}\pth{\mathbf{M}_1}\\
\mathrm{det}\pth{\mathbf{M}_1}\\
\mathrm{det}\pth{\mathbf{M}_3}\\
\mathrm{det}\pth{\mathbf{M}_3}\\
\end{bmatrix}
\end{align*}


\begin{minted}{cpp}
__m256 determinants11330022= Permute2F128<1, 0>(determinants00221133);
\end{minted} 
\begin{align*}
\mathrm{determinants11330022} 
&=
\begin{bmatrix}
\mathrm{det}\pth{\mathbf{M}_1}\\
\mathrm{det}\pth{\mathbf{M}_1}\\
\mathrm{det}\pth{\mathbf{M}_3}\\
\mathrm{det}\pth{\mathbf{M}_3}\\
\mathrm{det}\pth{\mathbf{M}_0}\\
\mathrm{det}\pth{\mathbf{M}_0}\\
\mathrm{det}\pth{\mathbf{M}_2}\\
\mathrm{det}\pth{\mathbf{M}_2}\\
\end{bmatrix}
\end{align*}

\begin{minted}{cpp}
__m256 determinants 
    = Blend<0, 1, 0, 1, 1, 0, 1, 0>(determinants00221133, determinants11330022);
\end{minted} 
\begin{align*}
\mathrm{determinants} 
&=
\begin{bmatrix}
\mathrm{det}\pth{\mathbf{M}_0}\\
\mathrm{det}\pth{\mathbf{M}_1}\\
\mathrm{det}\pth{\mathbf{M}_2}\\
\mathrm{det}\pth{\mathbf{M}_3}\\
\mathrm{det}\pth{\mathbf{M}_0}\\
\mathrm{det}\pth{\mathbf{M}_1}\\
\mathrm{det}\pth{\mathbf{M}_2}\\
\mathrm{det}\pth{\mathbf{M}_3}\\
\end{bmatrix}
\end{align*}

Finally, the system's solution is calculated and returned.

\begin{minted}{cpp}
__m256 solution = _mm_div(determinants, detA);

return Vec4fSSE<true>(_mm256_castps256_ps128(solution));
\end{minted}


\vspace{1cm}
\textbf{Complete Function:}

\begin{minted}{cpp}
inline Vec4fSSE<true> Cramer(const Mat4fAVX& matA, const Vec4fSSE<true>& vecRhs)
{
    using namespace GDL::simd;

    const std::array<__m256, 2>& dataA = matA.DataAVX();
    const __m256& ab = dataA[0];
    const __m256& cd = dataA[1];


    // Calculate the components of the first 4 terms of det(a)
    __m256 abP1230 = Permute<1, 2, 3, 0>(ab);
    __m256 cdP1230 = Permute<1, 2, 3, 0>(cd);

    __m256 acbd03 = _mm_fmsub(ab, cdP1230, _mm_mul(cd, abP1230));
    __m256 acbd03N = Negate<0, 0, 0, 0, 1, 0, 1, 0>(acbd03);


    // Calculate the components of the last 2 terms of det(a)
    __m256 abP2323 = Permute<2, 3, 2, 3>(ab);
    __m256 cdP2323 = Permute<2, 3, 2, 3>(cd);
    __m256 acbd45 = _mm_fmsub(ab, cdP2323, _mm_mul(cd, abP2323));


    // Calculate determinant of A
    __m256 ac03bd45 = Blend<0, 0, 0, 0, 1, 1, 1, 1>(acbd03N, acbd45);
    __m256 bd03ac45 = Permute2F128<0, 1, 1, 0>(acbd03N, acbd45);
    __m256 bd03ac45P = Permute<2, 3, 0, 1, 1, 0, 2, 3>(bd03ac45);

    __m256 sums = DotProduct(ac03bd45, bd03ac45P);

    __m256 detA = _mm_add(sums, Permute2F128<1, 0>(sums));


    DEV_EXCEPTION(_mm_cvtsF(detA) == ApproxZero<F32>(10), 
                  "Singular matrix - system not solveable");

    const __m256 r = _mm256_insertf128_ps(_mm256_castps128_ps256(vecRhs.DataSSE()), 
                                                                 vecRhs.DataSSE(), 1);


    // Calculate first 4 products of all modified matrix determinants
    __m256 rP1230 = Permute<1, 2, 3, 0>(r);

    __m256 rcrd03 = _mm_fmsub(r, cdP1230, _mm_mul(rP1230, cd));
    __m256 arbr03 = _mm_fmsub(ab, rP1230, _mm_mul(r, abP1230));

    __m256 rcrd03N = Negate<0, 0, 0, 0, 1, 0, 1, 0>(rcrd03);
    __m256 arbr03N = Negate<0, 0, 0, 0, 1, 0, 1, 0>(arbr03);

    __m256 bdac03 = Permute2F128<1, 0>(acbd03N);
    __m256 bdac03P2301 = Permute<2, 3, 0, 1>(bdac03);

    __m256 products03M0M1 = _mm_mul(rcrd03N, bdac03P2301);
    __m256 products03M2M3 = _mm_mul(arbr03N, bdac03P2301);
    __m256 products03B0011 
        = Blend<0, 0, 1, 1, 0, 0, 1, 1>(products03M0M1, products03M2M3);
    __m256 products03B1100 
        = Blend<1, 1, 0, 0, 1, 1, 0, 0>(products03M0M1, products03M2M3);
    __m256 products03B1100P2301 = Permute<2, 3, 0, 1>(products03B1100);

    __m256 sum03 = _mm_add(products03B0011, products03B1100P2301);


    // Calculate last 2 products of all modified matrix determinants
    __m256 acbdB1100 = Blend<1, 1, 0, 0, 1, 1, 0, 0>(ab, cd);
    __m256 acbdB0011 = Blend<0, 0, 1, 1, 0, 0, 1, 1>(ab, cd);
    __m256 acbdB0011P2301 = Permute<2, 3, 0, 1>(acbdB0011);

    __m256 rP2301 = Permute<2, 3, 0, 1>(r);

    __m256 components45 = _mm_fmsub(acbdB0011P2301, r, _mm_mul(rP2301, acbdB1100));
    __m256 bdac45 = Permute2F128<1, 0>(acbd45);
    __m256 bdac45P1010 = Permute<1, 0, 1, 0>(bdac45);

    __m256 products45M = _mm_mul(components45, bdac45P1010);


    // Calculate and return solution
    __m256 sumsM = _mm_add(sum03, products45M);
    __m256 determinants00221133 = _mm_add(sumsM, Permute<1, 0, 3, 2>(sumsM));
    __m256 determinants11330022 = Permute2F128<1, 0>(determinants00221133);
    __m256 determinants = Blend<0, 1, 0, 1, 1, 0, 1, 0>(determinants00221133, 
                                                        determinants11330022);

    __m256 solution = _mm_div(determinants, detA);

    return Vec4fSSE<true>(_mm256_castps256_ps128(solution));
}
\end{minted}
\newpage
\section{Gaussian elimination}
\label{sec:gaussianElimination}
\subsection{Theory}

\subsubsection{Algorithm (without pivoting)}
The standard Gaussian elimination algorithm brings the matrix representation of the system that should be solved into upper triangle form:


\begin{align}
\label{eq:gauss3x3upperTriangle}
\begin{bmatrix}
a_{00}&a_{01}&a_{02}\\
0&a_{11}&a_{12}\\
0&0&a_{22}\\
\end{bmatrix}
\cdot
\begin{bmatrix}
x_0\\
x_1\\
x_2\\
\end{bmatrix}
=
\begin{bmatrix}
b_0\\
b_1\\
b_2\\
\end{bmatrix}
\end{align}

In this form, simple backward substitution can be used to determine all unknown values $x_0$, $x_1$ and $x_2$.
The procedure to determine the upper triangular form will be illustrated on a short example.
Consider the following system:

\begin{align}
\label{eq:gauss3x3unmodified}
\begin{bmatrix}
	2&4&8\\
	1&1&1\\
	3&4&5\\
\end{bmatrix}
\cdot
\begin{bmatrix}
x_0\\
x_1\\
x_2\\
\end{bmatrix}
=
\begin{bmatrix}
2\\
1\\
2\\
\end{bmatrix}
\end{align}

Its solution is:

\begin{align*}
\begin{bmatrix}
	x_0\\
	x_1\\
	x_2\\
\end{bmatrix}
=
\begin{bmatrix}
	3\\
	-3\\
	1\\
\end{bmatrix}
\end{align*}


Multiplying individual rows with a scalar and subtracting a multiple of a row from another one does not affect the system's solution.
Note, that not only the matrix must be modified by the chosen operation, but also the right hand side vector.


This can be used to modify the matrix in a way, that all values of the first column, except the first one, become zero.
In the system of \cref{eq:gauss3x3unmodified}, this can be achieved by multiplying the first row with $\frac{1}{2}$ and subtracting the result from the second one.
The same thing is done for the third row but with a factor of $\frac{3}{2}$.
Notice that the factors are calculated by dividing the first value of a row by the first element of the first row, which is also the first value of the main diagonal.
This yields:

\begin{align*}
\begin{bmatrix}
	 \textcolor{red}{2}
	&\textcolor{red}{4}
	&\textcolor{red}{8}\\
	 \left(1 - \textcolor{blue}{\frac{1}{2}} \cdot \textcolor{red}{2}\right)
	&\left(1 - \textcolor{blue}{\frac{1}{2}} \cdot \textcolor{red}{4}\right)
	&\left(1 - \textcolor{blue}{\frac{1}{2}} \cdot \textcolor{red}{8}\right)\\
	 \left(3 - \textcolor{blue}{\frac{3}{2}} \cdot \textcolor{red}{2}\right)
	&\left(4 - \textcolor{blue}{\frac{3}{2}} \cdot \textcolor{red}{4}\right)
	&\left(5 - \textcolor{blue}{\frac{3}{2}} \cdot \textcolor{red}{8}\right)\\
\end{bmatrix}
&\cdot
\begin{bmatrix}
x_0\\
x_1\\
x_2\\
\end{bmatrix}
=
\begin{bmatrix}
\textcolor{red}{2}\\
\left(1 - \textcolor{blue}{\frac{1}{2}} \cdot \textcolor{red}{2}\right)\\
\left(2 - \textcolor{blue}{\frac{3}{2}} \cdot \textcolor{red}{2}\right)\\
\end{bmatrix}
\\
\begin{bmatrix}
2&4&8\\
0&-1&-3\\
0&-2&-7\\
\end{bmatrix}
&\cdot
\begin{bmatrix}
x_0\\
x_1\\
x_2\\
\end{bmatrix}
=
\begin{bmatrix}
2\\
0\\
-1\\
\end{bmatrix}
\end{align*}

The upper diagonal form can now be obtained by repeating this procedure for the subsystem that remains if the rows and columns above and to the left of the second value of the main diagonal are ignored.

\begin{align}
\nonumber
\begin{bmatrix}
	2&
	4&
	8\\
	0&
	-1&
	-3\\
	\left(0 - \textcolor{blue}{\frac{-2}{-1}} \cdot \textcolor{red}{0} \right)&
	\left(-2 - \textcolor{blue}{\frac{-2}{-1}} \cdot \textcolor{red}{-1} \right)&
	\left(-7 - \textcolor{blue}{\frac{-2}{-1}} \cdot \textcolor{red}{-3} \right)\\
\end{bmatrix}
&\cdot
\begin{bmatrix}
	x_0\\
	x_1\\
	x_2\\
\end{bmatrix}
=
\begin{bmatrix}
	2\\
	0\\
	\left(-1 - \textcolor{blue}{\frac{-2}{-1}} \cdot \textcolor{red}{0} \right)\\
\end{bmatrix}
\\
\label{eq:gauss3x3triangular}
\begin{bmatrix}
2&4&8\\
0&-1&-3\\
0&0&-1\\
\end{bmatrix}
&\cdot
\begin{bmatrix}
x_0\\
x_1\\
x_2\\
\end{bmatrix}
=
\begin{bmatrix}
2\\
0\\
-1\\
\end{bmatrix}
\end{align}

If the matrix is larger than in the example, the algorithm would continue with the subsystem that remains after removing the columns and rows  above and to the left of the third main diagonal value and so on until the upper triangular form is achieved.

Now the unknown values can be determined by backward substitution.
The value of $x_2$ follows from the last row: 


\begin{align*}
 x_2 = 1
\end{align*}

The second row yields:

\begin{align*}
-1x_1 - 3x_2 &= 0\\
x_1 &= -3x_2\\
x_1 &= -3
\end{align*}

The first row gives:
\begin{align*}
2x_0 + 4x_1 + 8x_2 &= 2\\
2x_0  &= 2 -  4x_1 - 8x_2\\
x_0  &= 1 -  2x_1 - 4x_2 \\
x_0  &= 1 -  \left(2\cdot-3\right) - \left(4\cdot 1\right) \\
x_0  &= 3
\end{align*}

\subsubsection{Gauss-Jordan algorithm}

The Gauss-Jordan algorithm works quite similar to the standard algorithm shown in the previous section.
The major difference is, that instead of an upper triangular matrix, it produces a diagonal matrix.
This makes the backward substitution step obsolete.
The first step is identical in both algorithms, but in the second step, the first row is not ignored:

\begin{align*}
\begin{bmatrix}
\left(2 - \textcolor{blue}{\frac{4}{-1}} \cdot \textcolor{red}{0} \right)&
\left(4 - \textcolor{blue}{\frac{4}{-1}} \cdot \textcolor{red}{-1} \right)&
\left(8 - \textcolor{blue}{\frac{4}{-1}} \cdot \textcolor{red}{-3} \right)\\
0&
-1&
-3\\
\left(0 - \textcolor{blue}{\frac{-2}{-1}} \cdot \textcolor{red}{0} \right)&
\left(-2 - \textcolor{blue}{\frac{-2}{-1}} \cdot \textcolor{red}{-1} \right)&
\left(-7 - \textcolor{blue}{\frac{-2}{-1}} \cdot \textcolor{red}{-3} \right)\\
\end{bmatrix}
&\cdot
\begin{bmatrix}
x_0\\
x_1\\
x_2\\
\end{bmatrix}
=
\begin{bmatrix}
\left(2 - \textcolor{blue}{\frac{4}{-1}} \cdot \textcolor{red}{0} \right)\\
0\\
\left(-1 - \textcolor{blue}{\frac{-2}{-1}} \cdot \textcolor{red}{0} \right)\\
\end{bmatrix}
\\
\begin{bmatrix}
2&0&-4\\
0&-1&-3\\
0&0&-1\\
\end{bmatrix}
&\cdot
\begin{bmatrix}
x_0\\
x_1\\
x_2\\
\end{bmatrix}
=
\begin{bmatrix}
2\\
0\\
-1\\
\end{bmatrix}
\end{align*}

In contrast to the general algorithm, an extra step at the end is needed to get the diagonal form.
The procedure remains the same with the only difference to previous steps being the fact that there is no more row below the pivot element:

\begin{align*}
\begin{bmatrix}
\left(2 - \textcolor{blue}{\frac{-4}{-1}} \cdot \textcolor{red}{0} \right)&
\left(0 - \textcolor{blue}{\frac{-4}{-1}} \cdot \textcolor{red}{0} \right)&
\left(-4 - \textcolor{blue}{\frac{-4}{-1}} \cdot \textcolor{red}{-1} \right)\\
\left(0 - \textcolor{blue}{\frac{-3}{-1}} \cdot \textcolor{red}{0} \right)&
\left(-1 - \textcolor{blue}{\frac{-3}{-1}} \cdot \textcolor{red}{0} \right)&
\left(-3 - \textcolor{blue}{\frac{-3}{-1}} \cdot \textcolor{red}{-1} \right)\\
0&
0&
-1\\
\end{bmatrix}
&\cdot
\begin{bmatrix}
x_0\\
x_1\\
x_2\\
\end{bmatrix}
=
\begin{bmatrix}
\left(2 - \textcolor{blue}{\frac{-4}{-1}} \cdot \textcolor{red}{-1} \right)\\
\left(0 - \textcolor{blue}{\frac{-3}{-1}} \cdot \textcolor{red}{-1} \right)\\
-1\\
\end{bmatrix}
\\
\begin{bmatrix}
2&0&0\\
0&-1&0\\
0&0&-1\\
\end{bmatrix}
&\cdot
\begin{bmatrix}
x_0\\
x_1\\
x_2\\
\end{bmatrix}
=
\begin{bmatrix}
6\\
3\\
-1\\
\end{bmatrix}
\end{align*}

If the right hand side values are divided by the corresponding matrix main diagonal value, the same result as in the previous section is obtained.


\subsubsection{Pivoting}


\subsection{Optimizations}

\subsection{Implementation}
\subsubsection{Serial version}
\subsubsection{Vectorized version}

\section{LU decomposition}
\label{sec:LU}

\section{LLT Cholesky decomposition}
\label{sec:lltCholeskyDecomposition}

\section{LDL decomposition}
\label{sec:LDL}

\section{Inverse Matrix}
\label{sec:InverseMatrix}

\section{Full factorization}
\label{sec:fullFactorization}




\section{Solver selection}
\label{sec:solverSelection}

\end{document}
