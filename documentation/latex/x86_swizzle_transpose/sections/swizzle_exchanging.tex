\subsection{Exchange}

The \cppInline{Exchange} function modifies the passed registers directly (pass by reference) and does not return a new register.
It exchanges two values between both the registers.
The indices of the corresponding data elements are specified as template parameters.
Note that the cost of this operation is significantly higher if the exchanged values are not in the same register lane. 

\subsubsection*{Example:}
\begin{minted}{cpp}
Exchange<1, 6>(a, b)
\end{minted}

\begin{tikzpicture}[>=stealth,thick,baseline, every node/.style={text height=2ex,text width=1em}]
\matrix [matrix of math nodes,
left delimiter={[},
right delimiter={]},
](A) at (0,0){ 
	|[fill=red!20]|\mathbf{a_{0}}\\
	|[fill=red!40]|\mathbf{a_{1}}\\  
	|[fill=red!60]|\mathbf{a_{2}}\\
	|[fill=red!80]|\mathbf{a_{3}}\\
	\textbf{---}\\
	|[fill=blue!15]|\mathbf{a_{4}}\\
	|[fill=blue!30]|\mathbf{a_{5}}\\
	|[fill=blue!55]|\mathbf{a_{6}}\\
	|[fill=blue!70]|\mathbf{a_{7}}\\
};

\matrix [matrix of math nodes,
left delimiter={[},
right delimiter={]},
](B) at (5,0){ 
	|[fill=green!20]|\mathbf{b_{0}}\\
	|[fill=green!40]|\mathbf{b_{1}}\\  
	|[fill=green!60]|\mathbf{b_{2}}\\
	|[fill=green!80]|\mathbf{b_{3}}\\
	\textbf{---}\\
	|[fill=black!15]|\mathbf{b_{4}}\\
	|[fill=black!30]|\mathbf{b_{5}}\\
	|[fill=black!55]|\mathbf{b_{6}}\\
	|[fill=black!70]|\mathbf{b_{7}}\\
};

\draw[<->,color=black]([xshift= 12pt]A-2-1.east)-- ([xshift=-12pt]B-8-1.west);
\end{tikzpicture}
\hspace{1cm}
result: 
\begin{tikzpicture}[>=stealth,thick,baseline, every node/.style={text height=2ex,text width=1em}]
\matrix [matrix of math nodes,
left delimiter={[},
right delimiter={]},
](A) at (0,0){ 
	|[fill=red!20]|\mathbf{a_{0}}\\
	|[fill=black!55]|\mathbf{a_{1}}\\  
	|[fill=red!60]|\mathbf{a_{2}}\\
	|[fill=red!80]|\mathbf{a_{3}}\\
	\textbf{---}\\
	|[fill=blue!15]|\mathbf{a_{4}}\\
	|[fill=blue!30]|\mathbf{a_{5}}\\
	|[fill=blue!55]|\mathbf{a_{6}}\\
	|[fill=blue!70]|\mathbf{a_{7}}\\
};

\matrix [matrix of math nodes,
left delimiter={[},
right delimiter={]},
](B) at (2,0){ 
	|[fill=green!20]|\mathbf{b_{0}}\\
	|[fill=green!40]|\mathbf{b_{1}}\\  
	|[fill=green!60]|\mathbf{b_{2}}\\
	|[fill=green!80]|\mathbf{b_{3}}\\
	\textbf{---}\\
	|[fill=black!15]|\mathbf{b_{4}}\\
	|[fill=black!30]|\mathbf{b_{5}}\\
	|[fill=red!40]|\mathbf{b_{6}}\\
	|[fill=black!70]|\mathbf{b_{7}}\\
};
\end{tikzpicture}