\subsection{Broadcast functions}

Broadcast operations copy a single value (per lane) from a source register to all data elements (per lane) of the result register.  

\subsubsection{Broadcast}

The \cppInline{Broadcast} function broadcast one value per lane.
The position inside the lane of the data element that should be broadcasted is passed as template parameter.
For multi-lane registers it is also possible to select different positions per lane by providing as many template parameters as the register has lanes.

\subsubsection*{Example 1:}
\begin{minted}{cpp}
__m256 b = Broadcast<2>(a)
\end{minted}

\begin{tikzpicture}[>=stealth,thick,baseline, every node/.style={text height=2ex,text width=1em}]
\plotSwizzleTwoRegisters{2}{2}{2}{2}{6}{6}{6}{6}
\draw[->,color=black ]([xshift= 12pt]A-3-1.east)-- ([xshift=-12pt]B-1-1.west);
\draw[->,color=black ]([xshift= 12pt]A-3-1.east)-- ([xshift=-12pt]B-2-1.west);
\draw[->,color=black ]([xshift= 12pt]A-3-1.east)-- ([xshift=-12pt]B-3-1.west);
\draw[->,color=black ]([xshift= 12pt]A-3-1.east)-- ([xshift=-12pt]B-4-1.west);
\draw[->,color=black ]([xshift= 12pt]A-8-1.east)-- ([xshift=-12pt]B-6-1.west);
\draw[->,color=black ]([xshift= 12pt]A-8-1.east)-- ([xshift=-12pt]B-7-1.west);
\draw[->,color=black ]([xshift= 12pt]A-8-1.east)-- ([xshift=-12pt]B-8-1.west);
\draw[->,color=black ]([xshift= 12pt]A-8-1.east)-- ([xshift=-12pt]B-9-1.west);
\end{tikzpicture}


\subsubsection*{Example 2:}
\begin{minted}{cpp}
__m256 b = Broadcast<0, 3>(a)
\end{minted}

\begin{tikzpicture}[>=stealth,thick,baseline, every node/.style={text height=2ex,text width=1em}]
\plotSwizzleTwoRegisters{0}{0}{0}{0}{7}{7}{7}{7}
\draw[->,color=black ]([xshift= 12pt]A-1-1.east)-- ([xshift=-12pt]B-1-1.west);
\draw[->,color=black ]([xshift= 12pt]A-1-1.east)-- ([xshift=-12pt]B-2-1.west);
\draw[->,color=black ]([xshift= 12pt]A-1-1.east)-- ([xshift=-12pt]B-3-1.west);
\draw[->,color=black ]([xshift= 12pt]A-1-1.east)-- ([xshift=-12pt]B-4-1.west);
\draw[->,color=black ]([xshift= 12pt]A-9-1.east)-- ([xshift=-12pt]B-6-1.west);
\draw[->,color=black ]([xshift= 12pt]A-9-1.east)-- ([xshift=-12pt]B-7-1.west);
\draw[->,color=black ]([xshift= 12pt]A-9-1.east)-- ([xshift=-12pt]B-8-1.west);
\draw[->,color=black ]([xshift= 12pt]A-9-1.east)-- ([xshift=-12pt]B-9-1.west);
\end{tikzpicture}



\subsubsection{BroadcastAcrossLanes}

For single lane registers, `BroadcastAcrossLanes` behaves the same as `Broadcast`.
For multi-lane registers, a single value selected by a template parameter is written to all data elements of the result register. 
Keep in mind, that broadcasting across lane boundaries is a rather expensive operation and should only be used if it is really necessary.

\subsubsection*{Example:}
\begin{minted}{cpp}
__m256 b = BroadcastAcrossLanes<5>(a)
\end{minted}

\begin{tikzpicture}[>=stealth,thick,baseline, every node/.style={text height=2ex,text width=1em}]
\plotSwizzleTwoRegisters{5}{5}{5}{5}{5}{5}{5}{5}
\draw[->,color=black ]([xshift= 12pt]A-7-1.east)-- ([xshift=-12pt]B-1-1.west);
\draw[->,color=black ]([xshift= 12pt]A-7-1.east)-- ([xshift=-12pt]B-2-1.west);
\draw[->,color=black ]([xshift= 12pt]A-7-1.east)-- ([xshift=-12pt]B-3-1.west);
\draw[->,color=black ]([xshift= 12pt]A-7-1.east)-- ([xshift=-12pt]B-4-1.west);
\draw[->,color=black ]([xshift= 12pt]A-7-1.east)-- ([xshift=-12pt]B-6-1.west);
\draw[->,color=black ]([xshift= 12pt]A-7-1.east)-- ([xshift=-12pt]B-7-1.west);
\draw[->,color=black ]([xshift= 12pt]A-7-1.east)-- ([xshift=-12pt]B-8-1.west);
\draw[->,color=black ]([xshift= 12pt]A-7-1.east)-- ([xshift=-12pt]B-9-1.west);
\end{tikzpicture}