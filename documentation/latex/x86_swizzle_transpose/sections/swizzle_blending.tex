\subsection{Blend functions}

A blend operation conditionally copies the data elements from either the first or second input registers to the same position inside a result register. 
The underlying x86 intrinsic function are \cppInline{_mm_blend_ps}, \cppInline{_mm_blend_pd}, \cppInline{_mm256_blend_ps} and \cppInline{_mm256_blend_pd}.
These functions need an integer as parameter that controls the value selection.
Several functions that calculate the control integer value based on chosen template parameters are implemented.
So it is not necessary to determine the correct integer value yourself.

On current x86 CPU architectures, blending is the most efficient swizzle operation, since multiple blends can be performed during a single CPU cycle.

\subsubsection{Blend}

The \cppInline{Blend} function has as many template parameters as data elements.
Each value van either be 0 or 1.
If a template parameter is set to 0, the corresponding data element receives its value from the first source register.
If it is 1, the value is taken from the second input register.

\vspace{1cm}
\begin{minipage}{\linewidth}
\subsubsection*{Example:}
\begin{minted}{cpp}
__m256 c = Blend<0, 1, 1, 0, 1, 0, 0, 0>(a, b)
\end{minted}

\begin{tikzpicture}[>=stealth,thick,baseline, every node/.style={text height=2ex,text width=1em}]
\plotSwizzleThreeRegisters{\colorA{0}}{\colorB{1}}{\colorB{2}}{\colorA{3}}{\colorB{4}}{\colorA{5}}{\colorA{6}}{\colorA{7}}

\draw[->,color=black ]([xshift= 12pt]A-1-1.east)-- ([xshift=-12pt]C-1-1.west);
\draw[->,color=black]([xshift=-12pt]B-2-1.west)-- ([xshift= 12pt]C-2-1.east);
\draw[->,color=black]([xshift=-12pt]B-3-1.west)-- ([xshift= 12pt]C-3-1.east);
\draw[->,color=black ]([xshift= 12pt]A-4-1.east)-- ([xshift=-12pt]C-4-1.west);
\draw[->,color=black]([xshift=-12pt]B-6-1.west)-- ([xshift= 12pt]C-6-1.east);
\draw[->,color=black ]([xshift= 12pt]A-7-1.east)-- ([xshift=-12pt]C-7-1.west);
\draw[->,color=black ]([xshift= 12pt]A-8-1.east)-- ([xshift=-12pt]C-8-1.west);
\draw[->,color=black ]([xshift= 12pt]A-9-1.east)-- ([xshift=-12pt]C-9-1.west);
\end{tikzpicture}
\end{minipage}


\subsubsection{BlendIndex}

The \cppInline{BlendIndex} function has only a single template parameter independent of the used register types' size.
It specifies the index of a single data element that is taken from the second source register. 
All other values are taken from the first one.

\vspace{1cm}
\begin{minipage}{\linewidth}
\subsubsection*{Example:}
\begin{minted}{cpp}
__m256 c = BlendIndex<5>(a, b)
\end{minted}


\begin{tikzpicture}[>=stealth,thick,baseline, every node/.style={text height=2ex,text width=1em}]
\plotSwizzleThreeRegisters{\colorA{0}}{\colorA{1}}{\colorA{2}}{\colorA{3}}{\colorA{4}}{\colorB{5}}{\colorA{6}}{\colorA{7}}

\draw[->,color=black ]([xshift= 12pt]A-1-1.east)-- ([xshift=-12pt]C-1-1.west);
\draw[->,color=black ]([xshift= 12pt]A-2-1.east)-- ([xshift=-12pt]C-2-1.west);
\draw[->,color=black ]([xshift= 12pt]A-3-1.east)-- ([xshift=-12pt]C-3-1.west);
\draw[->,color=black ]([xshift= 12pt]A-4-1.east)-- ([xshift=-12pt]C-4-1.west);
\draw[->,color=black ]([xshift= 12pt]A-6-1.east)-- ([xshift=-12pt]C-6-1.west);
\draw[->,color=black]([xshift=-12pt]B-7-1.west)-- ([xshift= 12pt]C-7-1.east);
\draw[->,color=black ]([xshift= 12pt]A-8-1.east)-- ([xshift=-12pt]C-8-1.west);
\draw[->,color=black ]([xshift= 12pt]A-9-1.east)-- ([xshift=-12pt]C-9-1.west);
\end{tikzpicture}
\end{minipage}


\subsubsection{BlendAboveIndex}

The \cppInline{BlendIndex} function has only a single template parameter independent of the used register types' size.
Data elements with a higher index than the specified value are taken from the second source register. 
All other values are taken from the first one.

\vspace{1cm}
\begin{minipage}{\linewidth}
\subsubsection*{Example:}
\begin{minted}{cpp}
__m256 c = BlendAboveIndex<5>(a, b)
\end{minted}

\begin{tikzpicture}[>=stealth,thick,baseline, every node/.style={text height=2ex,text width=1em}]
\plotSwizzleThreeRegisters{\colorA{0}}{\colorA{1}}{\colorA{2}}{\colorA{3}}{\colorA{4}}{\colorA{5}}{\colorB{6}}{\colorB{7}}

\draw[->,color=black ]([xshift= 12pt]A-1-1.east)-- ([xshift=-12pt]C-1-1.west);
\draw[->,color=black ]([xshift= 12pt]A-2-1.east)-- ([xshift=-12pt]C-2-1.west);
\draw[->,color=black ]([xshift= 12pt]A-3-1.east)-- ([xshift=-12pt]C-3-1.west);
\draw[->,color=black ]([xshift= 12pt]A-4-1.east)-- ([xshift=-12pt]C-4-1.west);
\draw[->,color=black ]([xshift= 12pt]A-6-1.east)-- ([xshift=-12pt]C-6-1.west);
\draw[->,color=black ]([xshift= 12pt]A-7-1.east)-- ([xshift=-12pt]C-7-1.west);
\draw[->,color=black]([xshift=-12pt]B-8-1.west)-- ([xshift= 12pt]C-8-1.east);
\draw[->,color=black]([xshift=-12pt]B-9-1.west)-- ([xshift= 12pt]C-9-1.east);
\end{tikzpicture}
\end{minipage}



\subsubsection{BlendBelowIndex}

The \cppInline{BlendIndex} function has only a single template parameter independent of the used register types' size.
Data elements with a lower index than the specified value are taken from the second source register. 
All other values are taken from the first one.

\vspace{1cm}
\begin{minipage}{\linewidth}
\subsubsection*{Example:}
\begin{minted}{cpp}
__m256 c = BlendBelowIndex<5>(a, b)
\end{minted}

\begin{tikzpicture}[>=stealth,thick,baseline, every node/.style={text height=2ex,text width=1em}]
\plotSwizzleThreeRegisters{\colorB{0}}{\colorB{1}}{\colorB{2}}{\colorB{3}}{\colorB{4}}{\colorA{5}}{\colorA{6}}{\colorA{7}}

\draw[->,color=black]([xshift=-12pt]B-1-1.west)-- ([xshift= 12pt]C-1-1.east);
\draw[->,color=black]([xshift=-12pt]B-2-1.west)-- ([xshift= 12pt]C-2-1.east);
\draw[->,color=black]([xshift=-12pt]B-3-1.west)-- ([xshift= 12pt]C-3-1.east);
\draw[->,color=black]([xshift=-12pt]B-4-1.west)-- ([xshift= 12pt]C-4-1.east);
\draw[->,color=black]([xshift=-12pt]B-6-1.west)-- ([xshift= 12pt]C-6-1.east);
\draw[->,color=black ]([xshift= 12pt]A-7-1.east)-- ([xshift=-12pt]C-7-1.west);
\draw[->,color=black ]([xshift= 12pt]A-8-1.east)-- ([xshift=-12pt]C-8-1.west);
\draw[->,color=black ]([xshift= 12pt]A-9-1.east)-- ([xshift=-12pt]C-9-1.west);
\end{tikzpicture}
\end{minipage}




\subsubsection{BlendInRange}

The \cppInline{BlendIndex} function has two template parameters independent of the used register types' size.
Data elements with indices equal or between those 2 values are taken from the second source register. 
All other are taken from the first one.

\vspace{1cm}
\begin{minipage}{\linewidth}
\subsubsection*{Example:}
\begin{minted}{cpp}
__m256 c = BlendInRange<3, 5>(a, b)
\end{minted}

\begin{tikzpicture}[>=stealth,thick,baseline, every node/.style={text height=2ex,text width=1em}]
\plotSwizzleThreeRegisters{\colorA{0}}{\colorA{1}}{\colorA{2}}{\colorB{3}}{\colorB{4}}{\colorB{5}}{\colorA{6}}{\colorA{7}}

\draw[->,color=black ]([xshift= 12pt]A-1-1.east)-- ([xshift=-12pt]C-1-1.west);
\draw[->,color=black ]([xshift= 12pt]A-2-1.east)-- ([xshift=-12pt]C-2-1.west);
\draw[->,color=black ]([xshift= 12pt]A-3-1.east)-- ([xshift=-12pt]C-3-1.west);
\draw[->,color=black]([xshift=-12pt]B-4-1.west)-- ([xshift= 12pt]C-4-1.east);
\draw[->,color=black]([xshift=-12pt]B-6-1.west)-- ([xshift= 12pt]C-6-1.east);
\draw[->,color=black]([xshift=-12pt]B-7-1.west)-- ([xshift= 12pt]C-7-1.east);
\draw[->,color=black ]([xshift= 12pt]A-8-1.east)-- ([xshift=-12pt]C-8-1.west);
\draw[->,color=black ]([xshift= 12pt]A-9-1.east)-- ([xshift=-12pt]C-9-1.west);
\end{tikzpicture}
\end{minipage}